\documentclass[12pt]{article}
\usepackage{a4wide, amsfonts, epsfig,bbding}

\begin{document}
\begin{center}
{\bf EMAT10001 Lecture 4.}\\[1cm]{} Conor Houghton 2013-10-16
\end{center}
\subsection*{Preface} 
These are outline notes for lecture 4; they are based on \emph{Number
  Theory with Computer Applications} by Ramanujachary Kumanduri and
Cristina Romero. This is an excellent book, but the material can be
found in many number theory and discrete mathematics books. As usual
there is a bounty of between 20p and \pounds 2 for errors, you can
tell me at the end of a lecture or email me at
\texttt{conor.houghton{@}bristol.ac.uk}. A manicule (\HandLeft{} or \HandRight) is used to
indicate that a proof, derivation or piece of material has been
omitted from the lecture but will be covered in the workshop.

\subsection*{Introduction}
There are two types of mathematicians, those who think of themselves
as discovering mathematics and those who think of themselves as
creating it. It is the sort of mathematics done by the second type of
mathematicians that we will look at over the next few lectures, in
particular, we will look at number theory and group theory. Our aim is
three fold:
\begin{itemize}
\item The mathematics we will look at is the basis of cryptography and is useful in a number of areas in computer science.
\item The concepts we will see occur, often in a more complicated way, across all sorts of mathematical constructions.
\item This area has some nice proofs, which will let us practise proving things before learning about it more precisely from Kerstin.
\end{itemize}

In the case of Boolean algebra we saw the \emph{and} table:\\
\begin{center}
\begin{tabular}{l|ll}
$\land$&0&1\\
\hline
0   &0&0\\
1   &0&1\\
\end{tabular}
\end{center}
which a lot like a multiplication table and the \emph{xor} table:\\
\begin{center}
\begin{tabular}{l|ll}
$\oplus$&0&1\\
\hline
0   &0&1\\
1   &1&0\\
\end{tabular}
\end{center}
which looks a bit like an addition table, except $1+1=2$ whereas
$1\oplus 1=0$. In fact, this could also be thought of as modular
addition. Here we are going to look at modular arithmetic and then at
group theory, which answers the question of when something \lq{}looks
a bit like an addition table\rq{}. Modular arithmetic is the basis of
the RSA encryption algorithm, it gives examples that are useful for
studying group theory, it is a good example for looking at equivalence
relations and it's a fun piece of mathematics.

\subsection*{Remainders}

We write
\begin{equation}
r=a \bmod b
\end{equation}
to mean that $r$ is the remainder when $a$ is divided by $b$. This is
ambiguous, a non-zero $r$ could be positive or negative; it is normal
to take $r$ non-negative. This is the same as saying
\begin{equation}
a=mb+r
\end{equation}
for some $m$ an integer, with $0\le r<b$. Hence
\begin{equation}
4=104 \bmod 50
\end{equation}
because $104=2\cdot 50+4$.\footnote{There are two simple Python programs
  on the website for you to practise modular arithmetic.} \HandRight{} It is easy
to check that 
\begin{equation}
[a+b\bmod{c}]=[[a \bmod{c}]+[b \bmod{c}] \bmod{c}]
\end{equation}
and
\begin{equation}
[ab\bmod{c}]=[[a \bmod{c}][b \bmod{c}] \bmod{c}]
\end{equation}
which helps when calculating remainders for large numbers, for
example, for numbers whose product might exceed the capacity of an
\texttt{int}.

We should also note that there is a theorem called The Division
Theorem that shows that this is all well defined, that is, given
integers $a$ and $b$ with $b\not=0$ there are unique integers $m$ and
$r$ such that $a=bm+r$ and $0\le r<|b|$. This isn't a hard theorem but
we are leaving it out for brevity.

If $0=a\bmod b$ then we say $b$ \emph{divides} $a$ and write
$b|a$. Obviously, this means $a=mb$ for some $m$. Hence $8|24$ and
$3|27$ but $8\not|27$. Notice also that $a|0$ since $0=0a$. \HandRight It is useful to list some basic properties as a lemma.\\
\\
\noindent \textbf{Lemma}. If $a$, $b$, $c$, $x$ and $y$ are positive integers
\begin{enumerate}
\item If $a|b$ and $x|y$ then $ax|by$.
\item If $a|b$ and $b|c$ then $a|c$.
\item If $a|b$ and $b\not=0$ then $a\le b$.
\item If $a|b$ and $a|c$ then $a|(bx+cy)$.
\end{enumerate}

As you probably know if $p>1$ has only one and itself as divisors then
$p$ is a \emph{prime} number. Thus, seven is a prime, eight is not. Perhaps
the most important thing about primes is that the set of integers is a \emph{unique factorization domain}, which means that every number can be written as
\begin{equation}
n=p_1p_2\ldots p_r
\end{equation}
where $p_1$, $p_2$ and so on up to $p_r$ are primes and up to the
order you write the primes in, there is only one way to do this. In this way
\begin{equation}
18=2\cdot 3 \cdot 3 =2 \cdot 3^2.
\end{equation}

\subsection*{Greatest Common Divisor}

Another important concept is the greatest common divisor of two
numbers $a$ and $b$, written gcd$(a,b)$ or just
$(a,b)$.\footnote{Because it is such a straight-forward notation,
  $(a,b)$ is used to mean very different things in different areas of
  mathematics, so be careful with it.} A \emph{common divisor} of $a$
and $b$ is a number $d$ which divides both, that is $d|a$ and $d|b$;
the \emph{greatest common divisor} is the largest such number. Hence
two is a common divisor of six and 24 but
\begin{equation}
(6,24)=6
\end{equation}
If $(a,b)=1$ we say $a$ and $b$ are \emph{co-prime}. 

There is an obvious, but slow way to find the greatest common divisor,
take for example 18 and 24. We know
\begin{eqnarray}
18&=&2\cdot 3\cdot 3\cr
24&=&2\cdot 2\cdot 2\cdot 3
\end{eqnarray}
so we look at the prime factors in common, one two and one three, so
$(18,24)=6$. There is a much quicker way, the Euclid algorithm, which
we will return to shortly, first though we want to prove a slightly
surprising property of the greatest common divisor; for any $a$ and
$b$ there are integers $m$ and $n$ such that
\begin{equation}
(a,b)=ma+nb
\end{equation}
This works for all integers $a$ and $b$ but for simplicity we assume $a$ and $b$ are non-negative. \HandRight We first need a simple lemma.\\
\\
\noindent \textbf{Lemma}. $(a,b)=(a-b,b)$.\\

Now the theorem itself.\\
\\
\noindent \textbf{Theorem}. For any non-negative integers $a$ and $b$ there are integers $m$ and $n$ such that $(a,b)=ma+nb$.\\
\noindent \textbf{Proof}: Write $d=(a,b)$. It is clear that $d|(ma+nb)$ for any
$m$ and $n$, the problem is showing that there is an $m$ and $n$ such
that $d=ma+nb$. Consider the set of all linear combinations
\begin{equation}
S=\{ma+nb|m,n\in\mathbf{Z}\}
\end{equation}
Let $c$ be the smallest positive integer in $S$, so $c>0$ and, if
$e\in S$ and $e>0$ then $e\ge c$. Since $c\in S$ there are $m_1$ and $n_1$ such that
\begin{equation}
c=m_1a+n_1b
\end{equation}
Now by choosing $m=1$ and $n=0$ or the other way around we can see
$c\le a$ and $c\le b$. Now from the division theorem we know
\begin{equation}
a=qc+r
\end{equation}
with $0\le r<c$ for some $q$. Turning this around
\begin{equation}
r=a-qc=a-q(m_1a+n_1b)=(1-qm_1)a-qn_1b
\end{equation}
so since this in the form $ma+nb$, with $m=1-qm_1$ and $n=-qn_1$,
$r\in S$. If $r\not=0$ then this $r$ is smaller than $S$,
contradicting the choice of $c$ as the smallest non-negative element
in $S$. Thus $r$ must be zero, and hence $a=qc$ so $c|a$. The same
method can be used to show $c|b$ and hence $c$ is a divisor of $a$ and
$b$. Since $c\in S$ we know $d|c$ and since $d$ is the greatest
divisor, this means $d=c$. $\square$

Related to the greatest common divisor is the \emph{Euler Totient function}:
\begin{equation}
\phi(a)=|\{b<a|(a,b)=1\}|
\end{equation}
so $\phi(a)$ is the number of numbers less than $a$ and co-prime with it.\HandLeft

\subsection*{Euclid algorithm}

The Euclid algorithm is an efficient algorithm for finding the
greatest common divisor of two numbers, it also allows us to find the $m$ and $n$ in the equation
\begin{equation}
(a,b)=ma+nb
\end{equation}
for some $m$ and $n$. It is perhaps surprising that it is far easier
to find the greatest common divisor than to factorize the numbers
involved. This is part of what makes public key cryptography work. The
algorithm is\\
\\
\noindent \textbf{Algorithm}. For positive integers $a$ and $b\not=0$ 
\begin{enumerate}
\item Set $x=a$ and $y=b$.
\item If $y=0$ then $x$ is the answer.
\item Set $r=x\bmod y$ and then let $x=y$ and $y=r$ and return to 2.
\end{enumerate}

The algorithm works because $(a,b)=(a-b,b)$, the step $r=x\mbox y$ is
replacing $x$ with $x-my$ for some $y$. It also swaps using
$(a,b)=(b,a)$ so that the numbers keep getting lower.

As an example consider finding (56,106). 
\begin{eqnarray}
50&=&106\bmod 56\cr
6 &=&56 \bmod 50\cr
2 &=&50 \bmod 6\cr
 0 &=&6 \bmod 2
\end{eqnarray}
so $(56,106)=2$. Further, working backwards through the modular divisions, from the first non-zero modular equation
\begin{equation}
2 =50-8*6
\end{equation}
Now the next one tells us that $6=56-50$, so substituting that back in gives
\begin{equation}
2 =50-8(56-50)=9\cdot 50 -8\cdot 56
\end{equation}
and $50=106-56$ gives
\begin{equation}
2 = 9\cdot (106-56) - 8\cdot 56 = 9\cdot 106 -17 \cdot 56
\end{equation}

\subsection*{Modular arithmetic}

The idea behind modular arithmetic is to use the idea of remainders
and so on and to turn it into an actual arithmetic system. It relies
on \emph{congruence}. If $a$, $b$ and $c$ are integers, we say that
$a$ \emph{is congruent to } $b$ \emph{modulo} $m$ if $m|(a-b)$. This
is written $a\equiv b\pmod m$. There are different ways of thinking
about this, we could say $a$ is the same as $b$ up to a multiple of
$m$, or we could use the language of remainders we developed above and
say $a\equiv b \pmod m$ is $(a \bmod m)=(b \bmod m)$. Note the way the
use of mod differs, if $r=a\bmod m$ then $r$ is the remainder and
$0\le r<b$ whereas if $a\equiv b\pmod m$ then $a$ and $b$ are
conjugate, neither has to be less than $m$.

Congruence is an example of an \emph{equivalence relation}, an equivalence relation is any relationship between pairs of elements in a set that satisfies reflexivity, symmetry and transitivity. So, for a set $X$ the relationship $\sim$ is an equivalence relationship if 
\begin{enumerate}
\item Reflexivity: if $x\in X$ then $x\sim x$.
\item Symmetry: if $x$ and $y$ are in $X$ and $x\sim y$ then $y\sim x$.
\item Transitivity: if $x$, $y$ and $z$ are in X and $x\sim y$ and
  $y\sim z$ then $x\sim z$.
\end{enumerate}
Verifying that congruence is an equivalence relation is just a matter of checking each of those three properties.\HandLeft
\end{document}


