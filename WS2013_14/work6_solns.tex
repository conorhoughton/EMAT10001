\documentclass[12pt]{article}

\usepackage{a4wide, amsfonts, epsfig,bbold}

\begin{document}
\begin{center}
{\bf EMAT10001 Workshop Sheet 6, outline solutions.}\\[1cm]{} Conor Houghton 2013-11-2
\end{center}

\subsubsection*{Work sheet}


\begin{enumerate}
\item Use Euler's theorem to calculate
\begin{equation}
3^{81}\pmod{100}
\end{equation}
\textbf{Solution: } So $100=2\cdot 2\cdot 5\cdot 5$ so 
\begin{equation}
\phi(100)=100\cdot\frac{1}{2}\frac{4}{5}=40
\end{equation}
Now $(3,100)=1$ so we can apply Euler's theorem and get
\begin{equation}
3^{40}\equiv 1 \pmod{100}
\end{equation}
and so
\begin{equation}
3^{81}\equiv 3 \pmod{100}
\end{equation}
\item An enemy organization has encrypted a message with the public
  key $p=111$ and $e=5$; since $p$ is three digits long the message
  blocks are all taken to be two digits, that is one character, long,
  with the simple translation of the alphabet into numbers from zero
  to 25 we have been using. The message is
  001101000081025032000109000021000 where each three digits
  corresponds to one letter. However, by choosing a public key $n$
  with less the 2048 bits the enemy organization has made itself
  vulnerable to a brute force decryption attack, that's your job.
  \textbf{Solution: } So first of all let's factorize $n=111$, it is
  $3\cdot 37$ so $\phi(111)=72$. Next we need to find the inverse of
  $e=5$, well
\begin{eqnarray}
72&=&14\cdot 5+2\cr
5&=&2\cdot 2+1
\end{eqnarray}
so $1=5-2\cdot 2=5-2\cdot(72-5\cdot 14)=29\cdot 5-2\cdot 72$ or $5^{-1}\equiv 29 \pmod{72}$. Now
\begin{equation}
1^{29}\equiv 1\pmod{111}
\end{equation}
so the first letter is \lq{}b\rq{},
\begin{equation}
101^{29}\equiv 11\pmod{111}
\end{equation}
so the second letter is \lq{}l\rq{}, the third letter is \lq{}a\rq{} and the fourth is 
\begin{equation}
101^{29}\equiv 12\pmod{111}
\end{equation}
or \lq{}m\rq{}. In fact, the message is \lq{}blamecanada\rq{}.

\item This is about encoding rather than decoding, choose two primes
  that multiply to give a three digit number, chose a exponent
  \lq{}e\rq{} and a short message to encode and encode it. Ideally you should decode it again afterwards.
\textbf{Solution: } Obviously this depends on the primes and so on. Lets do a quick example, say $n=11\cdot 17=187$. Now $\phi(143)=10\cdot 12=160$ so $e=7$ works as an exponent since $(7,160)=1$. Now, say the message is \lq{}waxon\rq{} which in numbers is $2200231413$. Now
\begin{eqnarray}
c(22)&\equiv& 22^{7}\equiv 44 \pmod{187}\cr
c(00)&\equiv& 0^{7}\equiv 0 \pmod{187}\cr
c(23)&\equiv& 23^{7}\equiv 133 \pmod{187}\cr
c(14)&\equiv& 14^{7}\equiv 108 \pmod{187}\cr
c(13)&\equiv& 23^{7}\equiv 106 \pmod{187}
\end{eqnarray}
so the cipher text is 044000133108106. To decode you would have to find $d\equiv 7^{-1}\pmod{160}$. In fact $160=7\cdot 22+6$ and $7=6+1$ so $1=7-6=7-(160-7\cdot 22)$ or
\begin{equation}
1=23\cdot 7-160
\end{equation}
or $7^{-1}\equiv 23\pmod{160}$ hence the decoding exponent is $23$. Applying it to the stuff above
\begin{equation}
44^{23}\equiv 22 \pmod{187}
\end{equation}
and so on.  
\end{enumerate}

\end{document}
