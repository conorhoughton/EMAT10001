\documentclass[12pt]{article}

\usepackage{a4wide, amsfonts, epsfig}


\begin{document}

\begin{enumerate}

\item Solve
\begin{equation}
\frac{df}{dt}=5(1-f)
\end{equation}
with $f(0)=0$.

\textbf{Answer}:
First, as always, make it more like a problem you already know, let 
\begin{equation}
g(x)=1-f(x)
\end{equation}
so $f(0)=0$ means $g(0)=1$ and $dg/dt=-df/dt$ so
\begin{equation}
\frac{dg}{dt}=-5g
\end{equation}
Now substitutte $g=Ae{rt}$ and find $r=-5$ so
\begin{equation}
g(t)=Ae^{-5t}
\end{equation}
and $g(0)=1$ gives $A=1$ and so the solution is 
\begin{equation}
f(t)=1-e^{-5t}
\end{equation}

\item Calculate a Taylor series for $\cos{x}$. 

\textbf{Answer}: So
\begin{equation}
\frac{d\cos{x}}{dt}=-\sin{t}
\end{equation}
and 
\begin{equation}
\frac{d^2\cos{x}}{dt^2}=-\cos{t}
\end{equation}
and then it goes around again with the extra minus and you are back
where you started. Now $\cos{0}=1$ and $\sin{0}=0$ so
\begin{equation}
\cos{t}=\sum_{n\mbox{ even}} (-1)^{n/2}\frac{1}{n!}
\end{equation}

\item For $f(x,y)=xy$ work out the directional
  derivative in the $(1,3)$ direction at $(1,1)$; don't forget to
  normalize the direction vector.

\textbf{Answer}: So let $\mathbf{u}=(1,3)$, it has length $\sqrt{10}$ so the corresponding normalized vector is $\hat{\mathbf{u}}=(1/\sqrt{10},3/\sqrt{10})$. Now
\begin{equation}
\mbox{div}\,f=(y,x)
\end{equation}
and hence $\hat{\mathbf{u}}\cdot\hat{\mathbf{u}}=(x+3y)/\sqrt{10}$ and at $(1,1)$ this gives $D_{\hat{\mathbf{u}}}f|_{(1,1)}=4/\sqrt{10}$.

\item The third differential operator is curl; it acts on vector
  fields to give another vector field. It is only defined in three
  dimensions and has quite a complicated form
\begin{equation}
\mbox{curl}\mathbf{v}=\left(\frac{\partial v_3}{\partial y}-\frac{\partial v_2}{\partial z},\frac{\partial v_1}{\partial z}-\frac{\partial v_3}{\partial x},\frac{\partial v_2}{\partial x}-\frac{\partial v_1}{\partial y}\right)
\end{equation}
Show grad$\,(\mbox{curl}\mathbf{v})=0$.

\textbf{Answer}:This is answered through sweat, toil and tears:
\begin{equation}
\mbox{grad}\,(\mbox{curl}\mathbf{v})=\frac{\partial^2 v_3}{\partial y\partial x}-\frac{\partial^2 v_2}{\partial z\partial x}+\frac{\partial^2 v_1}{\partial z\partial y}-\frac{\partial^2 v_3}{\partial x\partial y}+\frac{\partial^2 v_2}{\partial x\partial z}-\frac{\partial^2 v_1}{\partial y\partial z}
\end{equation}
and since the order of the partial differentiations doesn't matter,
this all cancels away to nothing.

\item Find the Fourier series for $\sin^3 t$; a quick way to do this is to regard it as a trigonometry problem, rather than a Fourier series problem, that is use the trigonometric identities to express it in terms of sines and cosines, rather than doing all the integrals: so start by writing $\sin^3 t=\sin^2 t \sin t$ and then write $\sin^2 t$ in terms of $\cos 2t$.

\textbf{Answer}: So this could be done with the integrals like any other Fourier series, but the funny thing is that it can also be done with trignometry.
\begin{equation}
\sin^3 t=\sin^2t \sin t
\end{equation}
Then using $\cos{2t}=\cos^2t - \sin^2 t$ and $1=\sin^2t +\cos^2t$ we have
\begin{equation}
\sin^2t = (1-\cos{2t})/2
\end{equation}
and so
\begin{equation}
\sin^3t=\frac{1}{2}\sin{t}-\frac{1}{2}\cos{2t}\sin{t}
\end{equation}
and now using your table of trignometric identities
\begin{equation}
\cos{2t}\sin{t}=(-\sin{t}+\sin{3t})/2
\end{equation}
or
\begin{equation}
\sin^3t=\frac{3}{4}\sin{t}-\frac{1}{4}\sin{3t}
\end{equation}
which is in the form of a Fourier series.






\end{enumerate}
\end{document}
