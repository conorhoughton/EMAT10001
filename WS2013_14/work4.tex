\documentclass[12pt]{article}

\usepackage{a4wide, amsfonts, epsfig}

\begin{document}
\begin{center}
{\bf EMAT10001 Workshop Sheet 4.}\\[1cm]{} Conor Houghton 2013-10-17
\end{center}
\subsubsection*{Introduction} 
This worksheet is about modular arithmetic, primes, greatest common
divisors, the Euler totient function and equivalence relations.

There is the usual bounty for errors and typos, 20p to \pounds 2
depending on how serious it is.

Some of these questions are taken from \emph{Number Theory with
  Computer Applications} by Ramanujachary Kumanduri and Cristina
Romero.

\subsubsection*{Useful facts}
\begin{itemize}
\item $r=a\bmod c$ means $a=mb+r$ and $0\le r<b$.
\item A prime number is one that has exactly two divisors.
\item Two numbers are co-prime, also called relatively prime, if $(a,b)=1$.
\item The Euclid algorithm is: set $x=a$ and $y=b$ where $a>b$, then, while $y!=0$
  set $r=x\bmod y$ and then let $x=y$ and $y=r$. If $y=0$ the answer
  is $x$.
\item  The \emph{Euler Totient function}:
\begin{equation}
\phi(a)=|\{b\le a|(a,b)=1\}|
\end{equation}
so $\phi(a)$ is the number of numbers less or equal $a$ and co-prime with it. So, for example, $\phi(6)=2$ because only one and five are less than six and co-prime with it.

\end{itemize}

\subsubsection*{Euclid algorithm, worked example}
I got the timing of my lecture a bit wrong and I never did an example
of the Euclid algorithm, so here is one. We want to find $(972,3834)$. Hence, set $x=3834$ and $y=972$. Now
\begin{equation}
3834 = 3*972 +918
\end{equation}
so now $x=972$ and $y=918$ and we get
\begin{equation}
972=918+54
\end{equation}
then with $x=918$ and $y=54$
\begin{equation}
918=17*54
\end{equation}
so in the next round $y$ would be equal zero so $(972,3834)=54$. We can also go backwards. 
\begin{equation}
54=972-918
\end{equation}
and $918=3834-3*972$ so
\begin{equation}
54=972-(3834-3*972)=4*972-3834
\end{equation}
which express $54=(972,3834)$ in the form $972n+3834m$.


\subsubsection*{Work sheet}

\begin{enumerate}

\item Think of two numbers $a$ and $b<a$, now calculate $r=a\bmod
  b$. Do this five times with bigger and bigger $a$. 


\item Prove that 
\begin{equation}
(a+b\bmod{c})=[(a \bmod{c})+(b \bmod{c}) \bmod{c}]
\end{equation}
and
\begin{equation}
(ab\bmod{c})=[(a \bmod{c})(b \bmod{c}) \bmod{c}]
\end{equation}



\item What number less than 100 has the most divisors?

\item If $p$ is a prime number how many divisors of $p^n$ are there?

\item If $a$, $b$, $c$, $x$ and $y$ are positive integers prove
\begin{enumerate}
\item If $a|b$ and $x|y$ then $ax|by$.
\item If $a|b$ and $b|c$ then $a|c$.
\item If $a|b$ and $b\not=0$ then $a<b$.
\item If $a|b$ and $a|c$ then $a|(bx+cy)$.
\end{enumerate}

\item Why isn't one a prime?



\item The next few questions are about the Euler totient function,
  which is defined in the useful facts section. First, think of a two
  digit number $a$ and work out $\phi(a)$. Do this five times.


\item If $p$ is prime what values of $a\le p^n$ have $(a,p^n)>1$?

\item If $p$ is prime, what is $\phi(p)$?

\item If $p$ is prime, what is $\phi(p^r)$?

\item If $(a,b)=1$ then $\phi(ab)=\phi(a)\phi(b)$; you can look this
  up, it is a consequence of the Chinese Remainder Theorem, which we
  aren't covering. Now, given the prime factorization of $n$ is
\begin{equation}
n=\prod p_i^{r_i}
\end{equation}
what is $\phi(n)$? Note that this means working out $\phi(n)$ is easy
if you can factorize $n$, for large numbers this is a big if! Use your
formula for some of the values of $\phi(n)$ you calculated above.

\item Prove $(a,b)=(a-b,b)$.

\item This question is about drawing a star without lifting your pencil
  off the paper. Imagine there are $n$ points equidistant around a
  circle. You want to draw a star by not lifting your pencil off the
  paper, it doesn't count as a star if you go from one point to the
  next, also, stars are symmetric, the angle at each of the points
  must be the same. So, for example, the star of David doesn't count
  because it involves two disconnected paths, but for $n=5$ there is
  one and $n=7$ there are two. How many different stars are there for
  $n$ points?

\item Think of two three digit numbers $a$ and $b$. Work out their greatest common divisor by first factorizing them. Repeat the calculation using the Euclid algorithm. Express the greatest common divisor in the form $ma+nb$ for integers $a$ and $b$.
Do this five times.

\item Prove that the slowest convergence for the Euclid algorithm occurs when the two numbers are consecutive terms in the Fibonacci sequence. 


\item Prove congruence is an equivalence relation.

\end{enumerate}

\subsubsection*{Exercise sheet}

The difference between the work sheet and the exercise sheet is that
the solutions to the exercise sheet won't be given and the problems
are designed to be more suited to working on on your own, though you
are free to discuss them in the work shop if you finish the work sheet
problems. Selected problems from the exercise sheet will be requested
as part of the continual assessment portfolio.

\begin{enumerate}

\item Determine the set of integers for which the number of divisors is odd. Make a general conjecture and prove your claim.

\item $n!=n(n-1)(n-2)\ldots 2\cdot 1$ so $5!=5\cdot 4\cdot 3\cdot
  2\cdot 1=120$. What are the prime factors of $12!$.

\item If $a$ and $b$ are integers does $a^2|b^3$ imply $a|b$? Prove or disprove.

\item Show $2|(n^2-n)$.

\item Show the numbers $6k+5$ and $7k+6$ are co-prime for every $k\ge 1$.

\item Write a program to implement the Euclid algorithm.

\item Extend your Euclid algorithm  so that it returns $(a,b)=ma+nb$.

\item Write a program to calculate primes using the Sieve of Eratosthenes.

\item Write a program to find the Euler Totient of numbers of modest size. 

\item Imagine you wanted to calculate $a^b\bmod c$ for large values of
  $a$ and $b$. The straight-forward approach of calculating $a^b$ and
  then taking its mod is inefficient and will overwhelm data types
  like \texttt{int}. The usual approach is to write $b$ in the binary form
\begin{equation}
b=b_0+2b_1+4b_2+8b_3+\ldots
\end{equation}
with all the $b_i$s one or zero. Now $a^2\bmod c$ is easy to work out
by squaring and modding, squaring that $(a^2)^2\bmod c$ gives $a^4$
and so on. Use this to write a program to work out $a^b\bmod c$ which
will work provided $a^2$ fits into \texttt{int}.

\end{enumerate}


\subsubsection*{Challenge}

There are copies of \emph{What is the name of this book?} and a kitkat
available to the first person to solve each of the projecteuler.net
problems 3, 50, 69 and 214; that is, there are four prizes, one for each
problem. Provide proof by sending a screen shot of the congratulations
page, I will announce on the website when each of the problems is solved.

\end{document}
