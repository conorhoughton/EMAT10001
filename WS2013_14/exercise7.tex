\documentclass[12pt]{article}

\usepackage{a4wide, amsfonts, epsfig,bbold}

\begin{document}
\begin{center}
{\bf EMAT10001 Workshop Sheet 7.}\\[1cm]{} Conor Houghton 2013-11-10
\end{center}
\subsubsection*{Introduction} 
There is the usual bounty for errors and typos, 20p to \pounds 2
depending on how serious it is.

\subsubsection*{Useful facts}
\begin{itemize}
\item Definition of a group: given a set $X$ and a map
\begin{eqnarray}
X\times X&\rightarrow& X\cr
(x,y)&\mapsto& x\cdot y
\end{eqnarray}
then $(X,\cdot)$ is a group if
\begin{enumerate}
\item Closure: if $x\in X$ and $y\in X$ then $x\cdot y\in X$.
\item Associativity: if $x$, $y$ and $z$ are all in $X$ then 
\begin{equation}
(x\cdot y)\cdot z=x\cdot(y\cdot z)
\end{equation}
\item Identity: there is an element $e\in X$ such that $x\cdot e=e\cdot x=x$ for all $x\in X$.
\item Inverse: for any element $x\in X$ there is another element
  $x^{-1}$ such that $x\cdot x^{-1}=x^{-1}\cdot x=e$.
\end{enumerate}
\end{itemize}

\subsubsection*{Some common mathematical notation}
\begin{itemize}
\item Some names for laws governing addition and multiplication.
\begin{enumerate}
\item Associative property: rough means you can move the brackets around, holds for both addition and multiplication. 
\begin{eqnarray}
a(bc)&=&(ab)c\cr
a+(b+c)&=&(a+b)+c
\end{eqnarray}
It doesn't hold for division: $(12/4)/3=3/3=1$ but $12/(4/3)=36/4=9$.
\item Distributive rule: the rule for getting rid of brackets when you have multiplication and addition
\begin{eqnarray}
a(b+c)&=&ab+ac\cr
(a+b)c&=&ac+bc
\end{eqnarray}
\item Abelian property: the order doesn't matter, holds for addition and multiplication.
\begin{eqnarray}
ab&=&ba\cr
a+b&=&b+a
\end{eqnarray}
Doesn't hold for division, or matrix multiplication or rotations about different axes.
\end{enumerate}
\item How to write maps: above we use a common notation for writing maps
\begin{eqnarray}
f:X&\rightarrow& Y\cr
  x&\mapsto& f(x)
\end{eqnarray}
means that $f$ maps elements in a set $X$ to elements in set $Y$ with $x$ going to $f(x)\in Y$. So, using this notation, if we were defining the floor function we might write
\begin{eqnarray}
\lfloor\cdot\rfloor: \textbf{R}&\rightarrow& \textbf{Z}\cr
x&\mapsto& \lfloor x\rfloor=\mbox{the integer you get by rounding down }x
\end{eqnarray}
\item The four element group we saw in lectures is called $\mathbf{Z}_4$, the one we discuss below, the Klein four-group is often called $V_4$, the $V$ stands for Vier, the German for four.
\end{itemize}


\subsubsection*{Exercise sheet}

\begin{enumerate}
\item The group $Z_2$ can be thought of as the multiplicative group formed by $\{[1],[2]\}$ modulo three. Write out the table.
\item The group $Z_2$ can be also be thought of as the additive group
  formed by $\{[0],[1]\}$ modulo two. Write out the table and show it
  is isomorphic to the table above.
\item Work out the group table for the rotational symmetries of an
  equilateral triangle.
\item A subgroup of a group is a subset of the group that is a group,
  the main thing to check is that the subset is closed. Now, using the
  notation in the lecture notes $\{e,a\}$ in the $Z_4$ group is not a
  subgroup since $a^2=c$ so it isn't closed. Can you find a $Z_2$
  subgroup of $Z_4$? What about $V_4$? It has three $Z_2$ subgroups.
\end{enumerate}

\subsubsection*{Further study}
\begin{itemize}
\item One nice story relates to wallpaper groups, they are the group of symmetries of a repeating two-dimensional pattern. It turns out there are only 17 of these; the Wikipedia article has illustrations of patterns with these different symmetries.\\ \texttt{http://en.wikipedia.org/wiki/Wallpaper\_group}
\item The \textsl{Futurama} episode \textsl{The Prisoner of Benda}
  involves group theory and one of the writers proved a theorem
  specifically to use in the episode.
\item The problem of finding all possible groups is one of the big problems of twentieth century mathematics, the eventual classification theorem is tens of thousands of pages long. See \texttt{http://en.wikipedia.org/wiki/Classification\_of\_finite\_simple\_groups}. Important work on this problem was done by John Conway who is known to computer scientists for inventing early cellular automaton called the Game of Life\\ \texttt{http://en.wikipedia.org/wiki/John\_Horton\_Conway}.
\end{itemize}

\subsubsection*{Challenge}
This week's \texttt{projecteuler.net} challenge: the usual sort of prize for the first two people to prove problems with numbers higher than 300.
\end{document}
