\documentclass[12pt]{article}

\usepackage{a4wide, amsfonts, epsfig}

\begin{document}
\begin{center}
{\bf EMAT10001 Workshop Sheet 14.}\\[1cm]{} Conor Houghton 2014-02-02
\end{center}
\subsubsection*{Introduction} 
This worksheet is about differentiation, the Taylor series and the Runge Kutta numerical algorithm. There is the usual bounty for errors and typos, 20p to \pounds 2
depending on how serious it is.

\subsubsection*{Useful facts}
\begin{itemize}
\item Differentiating the trigonometric functions:
\begin{eqnarray}
\frac{d}{dx}\cos{x}&=&-\sin{x}\cr
\frac{d}{dx}\sin{x}&=&\cos{x}
\end{eqnarray}
\item The Taylor expansion
\begin{equation}
f(t)=\sum_{n=0}^\infty\frac{1}{n!}\left.\frac{d^nf}{dt^n}\right|_{t=0}t^n
\end{equation}
This is the Taylor expansion around $t=0$, you can expand around any point
\begin{equation}
f(t)=\sum_{n=0}^\infty\frac{1}{n!}\left.\frac{d^nf}{dt^n}\right|_{t=t_0}(t-t_0)^n
\end{equation}
The Taylor expansion around $t=0$ is sometimes called the Maclaurin series. 
\item The Euler method for $\dot{y}=f(t,y)$ with $y(0)=y_0$
\begin{equation}
y_{n+1}=y_n+f(t_n,y_n)\delta t
\end{equation}
\item The midpoint method for $\dot{y}=f(t)$ with $y(0)=y_0$
\begin{eqnarray}
k_1&=&f(y_n)\cr
k_2&=&f(y_n+k_1\delta t/2)
\end{eqnarray}
then
\begin{equation}
y_{n+1}=y_n+k_2\delta t
\end{equation}
In the lecture notes this was referred to as the second order Runge
Kutta method. In fact, it is \textsl{a} second order Runge Kutta
method, we will see in the problems below that it is part of a larger family of such methods.
\item The fourth order Runge Kutta method.
\begin{equation}
\frac{dy}{dt}=f(t,y)
\end{equation}
Now
\begin{eqnarray}
k_1&=&f(t_n,y_n)\cr
k_2&=&f\left(t_n+\frac{1}{2}\delta,y_n+\frac{1}{2}\delta tk_1\right)\cr 
k_3&=&f\left(t_n+\frac{1}{2}\delta,y_n+\frac{1}{2}\delta tk_2\right)\cr 
k_4&=&f\left(t_n+\delta,y_n+\delta tk_2\right) 
\end{eqnarray}
and 
\begin{equation}
y_{n+1}=y_n+\frac{1}{6}(k_1+2k_2+2k_3+k_4)
\end{equation}
\item Notational note: 
\begin{equation}
\frac{d^2f}{dx^2}=\frac{d}{dx}\left(\frac{df}{dx}\right)
\end{equation}
\item There is some variation in how people write \lq{}the derivative of $f(x)$ at $x_0$\rq{}. If you are using the prime notation it is easy, this is $f'(x_0)$, using the $d/dx$ notation it is trickier, many people write 
\begin{equation}
f'(x_0)=\frac{df(x_0)}{dx}
\end{equation}
This is a poor notation in the sense that $f(x_0)$ is a constant, the variable $x$ has been replaced by a constant $x_0$, so the correct notation is 
\begin{equation}
f'(x_0)=\left.\frac{df(x)}{dx}\right|_{x=x_0}
\end{equation}
which shows that you differentiate first and then fix $x$. However, this notation is cumbersome, so although I use it the other one is acceptable; it is still a bit better to use
\begin{equation}
f'(x_0)=\frac{df}{dx}(x_0)
\end{equation}
to show that the differentiating is done first.
\item Another notational note: there are two common notations for the binomial coefficient $n$ choose $r$
\begin{equation}
_nC_r=\left(\begin{array}{c}n\\r\end{array}\right)=\frac{n!}{(n-r)!r!}
\end{equation}
\end{itemize}


\subsubsection*{Work sheet}

\begin{enumerate}


\item Revise differentiating  again!
\begin{enumerate}
\item Find $df/dx$ of $f(x)=\tan{x}$ using $\tan{x}=\sin{x}/\cos{x}$.
\item We know $\sin^2{x}+\cos^2{x}=1$, differentiate both sides of this equation.
\end{enumerate}

\item What is the Taylor expansion of $\tan{t}$ up to and include the $t^3$ term?

\item If $f(t)=\arctan{t}$ then 
\begin{equation}
\frac{df}{dt}=\frac{1}{1+t^2}
\end{equation}
This is derived using the chain rule and a trick called implicit differentiation. Basically let $y=\arctan{x}$ so $x=\tan{y}$, differentiate both sides with respect to $x$
\begin{equation}
1=\frac{dx}{dx}=\frac{d}{dx}\tan{y}=\frac{dy}{dx}\frac{d}{dy}\tan{y}
\end{equation}
and then do some messing to get $dy/dx$ in terms of $x$. Do that!


\item The Taylor expansion of $\arctan{x}$ is 
\begin{equation}
\arctan{x}=\sum_{n\,\mbox{odd}} (-1)^{(n-1)/2}\frac{x^n}{n}
\end{equation}
Check this as far as the $x^3$ term. [There is a very elegant derivation of this Taylor series, basically $1/(1+x)$ has a known Taylor expansion, up to a sign this is asked in the exercise sheet; this allows you to calculate the expansion of $1/(1+x^2)$ and from this you can read off all the derivative you need for the expansion of $\arctan{x}$, if you are feeling ambitious you can try to do this.]

\item What is $\arctan{1}$?

\item Use the Taylor expansion of $\arctan{1}$ to write down a formula
  for calculating $\pi$ and do the first six terms.


\item The series for $\arctan{1}$ is disastrously slow if you want to
  use it to calculate $\pi$; in fact getting ten decimal places
  correct requires 5,000,000,000 terms. However, there are other
  approaches to using the $\arctan{x}$ series. According to Wikipedia,
  in 1699, English mathematician Abraham Sharp used this series with
  $x=\sqrt{1/3} $ to compute $\pi$ to 71 digits, breaking the previous
  record of 39 digits, his record only stood for seven years when a
  new and faster converging series for $\pi$ was found; this was
\begin{equation}
\frac{\pi}{4}=4\arctan{\frac{1}{5}}-\arctan{\frac{1}{239}}
\end{equation}
You can prove this with clever use of trigonometric identities, you
shouldn't do that here, but you should use the formula to find an
approximation of $\pi$, take three terms for example. By the way,
there is a graph
at\\ \texttt{http://en.wikipedia.org/wiki/File:Record\_pi\_approximations.svg}\\ showing
how the number of known digits of $\pi$ has changed over history.


\item The binomial theorem tells us that
\begin{equation}
(1+x)^n=\sum_{r=0}^n \left(\begin{array}{c}n\\r\end{array}\right)x^r
\end{equation}
Prove this using the Taylor expansion.

\item Find the first three terms of the Taylor expansion of $\sqrt{1+x}$. Don't bother with this one if time is short since we've already done lots.

\item The general second order Runge Kutta method for $\dot{y}=f(y)$ is
\begin{eqnarray}
k_1&=&f(y_n)\cr
k_2&=&f(y_n+\alpha \delta t k_1)
\end{eqnarray}
and
\begin{equation}
y_{n+1}=y_n+\left[\left(1-\frac{1}{2\alpha}\right)k_1+\frac{1}{2\alpha}k_2\right]\delta t
\end{equation}
Show that this gives the Taylor series up to second order. Note that
$\alpha=1/2$ gives the midpoint method.

\end{enumerate}

\subsubsection*{Exercise sheet}

The difference between the work sheet and the exercise sheet is that
the solutions to the exercise sheet won't be given and the problems
are designed to be more suited to working on on your own, though you
are free to discuss them in the work shop if you finish the work sheet
problems. Selected problems from the exercise sheet will be requested
as part of the continual assessment portfolio.

\begin{enumerate}

\item Revise differentiating; find $df/dx$ of
\begin{enumerate}
\item $f(x)=1-\cos^2{x}$.
\item $f(x)=1/(1-x)$.
\end{enumerate}

\item Calculate a Taylor series for $\cos{x}$. 

\item Plot the behavior of different truncated Taylor series for
  $\cos{x}$ and compare them to $\cos{x}$ itself, truncate after one,
  two, four, eight and ten terms and plot over $-2\pi$ to $2\pi$.

\item We saw the natural log in last week's problem sheet, $\ln{x}=y$
  means $x=e^y$. now prove using the chain rule that
\begin{equation}
\frac{d}{dx}\ln{x}=\frac{1}{x}
\end{equation}

\item Calculate the Taylor series for $\ln{(1+t)}$ around $t=0$.

\item Calculate a Taylor series for $1/(1-x)$. Test the accuracy of the
  series after three terms for $x=0$, $x=0.5$, $x=0.75$ and $x=0.9$.

\item Write a program to implement the Euler method and the second and
  fourth order Runge-Kutta methods for solving differential equations
  for $\dot{y}=f(y)$; use these to solve the equation $\dot{P}=P(1-P)$
  mentioned last week and check the error of the various methods.

\end{enumerate}

\subsubsection*{Challenge}
First four to get onto level ten, that is have completed nine levels, of\\ 
\texttt{http://www.pythonchallenge.com/}\\ gets chocolate. Send a screenshot. 

\end{document}
