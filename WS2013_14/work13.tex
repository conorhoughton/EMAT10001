\documentclass[12pt]{article}

\usepackage{a4wide, amsfonts, epsfig}

\begin{document}
\begin{center}
{\bf EMAT10001 Workshop Sheet 13.}\\[1cm]{} Conor Houghton 2014-01-26
\end{center}
\subsubsection*{Introduction} 
This worksheet is about differentiation, the exponential function and
the growth equation. There is the usual bounty for errors and typos, 20p to \pounds 2
depending on how serious it is.

\subsubsection*{Useful facts}
\begin{itemize}
\item Recall the rules of differentiation
\begin{itemize}
\item Chain rule: 
\begin{equation}
\frac{d}{dx}f(g(x))=\frac{d}{dx}g(x)\frac{d}{dg}f(g)
\end{equation}
\item Product rule:
\begin{equation}
\frac{d}{dx}u(x)v(x)=u(x)\frac{d}{dx}v(x)+v(x)\frac{d}{dx}u(x)
\end{equation}
\item Powers:
\begin{equation}
\frac{d}{dx}x^n=nx^{n-1}
\end{equation}
\end{itemize}
\item The exponential function as the limit of continual compounding
\begin{equation}
\exp(x)=\lim_{n\rightarrow \infty}\left(1+\frac{x}{n}\right)^n
\end{equation}
\item The series for the exponential
\begin{equation}
\exp(x)=1+x+\frac{1}{2}x^2+\frac{1}{3!}x^3+\frac{1}{4!}x^4+\frac{1}{5!}x^5+\ldots
\end{equation}
\item Notational note: $df(t)/dt$ is often written $\dot{f}(t)$ and $df(x)/dx$ is often written $f'(x)$. The first notation is due to Leibnitz and is useful for remembering the chain rule and similar, the second is useful for writing the derivative at a particular point, say we want the derivative at $x=x_0$, where $x_0$ is a constant, we can write
\begin{equation}
f'(x_0)=\left.\frac{df}{dx}\right|_{x=x_0}
\end{equation}

\end{itemize}


\subsubsection*{Work sheet}

\begin{enumerate}

\item Revise differentiating; find $df/dx$ for the following.
\begin{enumerate}
\item $f(x)=1/x$
\item $f(x)=(1+x)^3$, do this one both by using the chain rule and by multiplying it out.
\item $f(x)=A\exp(\lambda x)$ where $A$ and $\lambda$ are constants.
\item $f(x)=x\exp(x)$
\item $f(x)=\exp(x^2)$
\item $f(x)=\exp(1/x)$
\item $f(x)=x\exp(1+x)$
\end{enumerate}

\item Using the power series, work out the value of $e$ to five decimal places.

\item By substituting in $f(t)=A\exp(\lambda t)$ solve 
\begin{equation}
\frac{d}{dt}f(t)=-2f(t)
\end{equation}
where $f(0)=5$. Substituting the $f(t)$ into the equation should give $\lambda$, fixing the initial condition, that is, the value of $f(0)$, should give $A$.

\item A breeding pair of rabbits are released on an island of endless
  grass. Each year each rabbit has five kits on average, these, in this
  approximation, grow up instantly. This rate is the average per rabbit,
  don't worry about the gender, the idea is that the doe have ten
  kits and the bucks none so the average is five. Thus, in this model
  if $P$ is the population of rabbits, the rate of increase is
  $dP/dt=5P$, ten kits for each member of the population and
  $P(0)=2$. What is the population after seven years.

\item Caesium-137 has a half-life of about 30 years; this means that a
  given Caesium-137 atom has a 0.0231 chance of decaying in a given
  year. To check this, write down a differential equation for the
  decay of Caesium-137 and solve it; check that the amount has halved
  after 30 years.

\item The natural log is the inverse of the exponential:
\begin{equation}
\ln e^x=x
\end{equation}
Check this using your calculator for a couple of values of $x$.

\item Find a formula relating the decay constant and the half life;
  the decay constant is the chance of an individual atom decaying in a
  given period of time, the 0.0231 in the example above.

\item I discover a new craze, everyday, on average, I manage to
  persuade one new person to start in on my craze, they do the same
  and so on and so on. How long before everyone in the world shares my
  craze assuming the growth rate stays the same. Why is this assumption absurd?

\item What is the $df/dx$ at $x=0$ for $f=\exp(-1/x)$, what about the second derivative 
\begin{equation}
\frac{d^2f}{dx^2}=\frac{d}{dx}\frac{df}{dx}
\end{equation}
at $x=0$. Can you guess what happens to higher derivatives at $x=0$?
To do this question you need to know that the exponential $\exp(-x)$
goes to zero as $x$ goes to infinity and, in fact, does so faster than
any polynomial! This means that there is a certain amount of guessing
involved since we haven't done limits, which makes this a hard question.

\end{enumerate}

\subsubsection*{Exercise sheet}

The difference between the work sheet and the exercise sheet is that
the solutions to the exercise sheet won't be given and the problems
are designed to be more suited to working on on your own, though you
are free to discuss them in the work shop if you finish the work sheet
problems. Selected problems from the exercise sheet will be requested
as part of the continual assessment portfolio.

\begin{enumerate}

\item Solve 
\begin{equation}
\frac{df}{dt}=5f
\end{equation}
with $f(0)=12$.

\item Differentiate 
\begin{equation}
f(x)=x^3e^{x^3}
\end{equation}

\item Solve
\begin{equation}
\frac{df}{dt}=5(1-f)
\end{equation}
with $f(0)=0$.

\item The growth equation is not a realistic model of growth if there
  is a finite resource the population requires, this might be food, or
  space, or available uninfected individuals. The Verhulst-Pearl
  equation is an alternative that includes a \textsl{carrying
    capacity} for the environment, growth depends not only on the
  population but the residual carrying capacity. A simple
  Velhulst-Pearl equation is
\begin{equation}
\frac{dP}{dt}=P(1-P)
\end{equation}
More complicated versions include constants which have been set to one here. Solving this equation is tricky, it involves direct integration and a partial fractions expansion. However, it is easier to check the solution is indeed a solution, the solution with $P(0)=1/2$ is
\begin{equation}
P=\frac{1}{1+\exp(-t)}
\end{equation}
Check this.
\end{enumerate}

\subsubsection*{Challenge}
First three to get onto level five, that is complete four levels, of \texttt{http://www.pythonchallenge.com/} gets chocolate. Send a screenshot.

\end{document}
