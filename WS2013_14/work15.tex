\documentclass[12pt]{article}

\usepackage{a4wide, amsfonts, epsfig}

\begin{document}
\begin{center}
{\bf EMAT10001 Workshop Sheet 15.}\\[1cm]{} Conor Houghton 2014-02-12
\end{center}
\subsubsection*{Introduction} 
Since there was no lecture this week this worksheet partly revizes
worksheets 13 and 14, it also includes some integration. There is the
usual bounty for errors and typos, 20p to \pounds 2 depending on how
serious it is. There is no exercize sheet for this week.

\subsubsection*{Useful facts}
\begin{itemize}
\item Differentiating the trigonometric functions:
\begin{eqnarray}
\frac{d}{dx}\cos{x}&=&-\sin{x}\cr
\frac{d}{dx}\sin{x}&=&\cos{x}
\end{eqnarray}
\item The Taylor expansion around $t=0$
\begin{equation}
f(t)=\sum_{n=0}^\infty\frac{1}{n!}\left.\frac{d^nf}{dt^n}\right|_{t=0}t^n
\end{equation}
\end{itemize}
\subsubsection*{Integration}
Integration is the inverse of differentiation:
\begin{equation}
\int_a^b \frac{df}{dt}dt=f(b)-f(a)
\end{equation}
and
\begin{equation}
\frac{d}{dt}\int_a^tfdt=f(t)
\end{equation}
The fundamental theorem of calculus tells us that the integral is also
the area under the curve, thus
\begin{equation}
A=\int_a^b f(t)dt
\end{equation}
is the area enclosed by $f(t)$ and the $t$ axis as $t$ goes from $t=a$
to $t=b$. 

You can work out some standard integrals from knowing that integration
is backwards diffentiation; these are often written without the limits,
so
\begin{equation}
\int t^ndt=\frac{t^{n+1}}{n+1}+C
\end{equation}
means
\begin{equation}
\frac{d}{dt}\left(\frac{t^{n+1}}{n+1}+C\right)=t^n
\end{equation}
no matter what constant value is given to $C$. This means
\begin{equation}
\int_a^b t^ndt=\frac{b^{n+1}}{n+1}-\frac{a^{n+1}}{n+1}
\end{equation}
and this is often written 
\begin{equation}
\int_a^b t^ndt=\left.\frac{t^{n+1}}{n+1}\right|_a^b
\end{equation}
So
\begin{eqnarray}
\int t^ndt&=&\frac{t^{n+1}}{n+1}+C\cr
\int e^tdt&=&e^t+C\cr
\int \sin{t}dt&=&-\cos{t}+C\cr
\int \cos{t}dt&=&-\sin{t}+C\cr
\int \frac{1}{t}dt&=&\ln{t}+C
\end{eqnarray}

Other integrals can often be done by a change of variables, for example, if 
\begin{equation}
I=\int e^{2t}dt
\end{equation}
let $s=2t$ and $ds/dt=2$ means $ds=2dt$, this manner of treating the
differential like a fraction and multiplying across by the $dt$
shouldn't make you think you can always get away with this sort of
thing, but there is a theorem that says you can here, so
\begin{equation}
I=\frac{1}{2}\int e^sds=\frac{1}{2}e^s+C=\frac{1}{2}e^{2t}+C
\end{equation}
Another, harder version would be
\begin{equation}
I=\int te^{t^2}dt
\end{equation}
Use $s=t^2$ so $ds/dt=2t$ or $ds=2tdt$ and
\begin{equation}
I=\int te^{t^2}dt=\frac{1}{2}\int e^{s}dt=\frac{1}{2}e^{t^2}+C
\end{equation}

If there are limits you need to change them too, so
\begin{equation}
I=\int_0^2 \exp{t/2}dt
\end{equation}
If you let $s=t/2$ then $2ds=dt$ and when $t=0$ then $s=0$ and when $t=2$ then $s=1$ so the integral becomes
\begin{equation}
I=2\int_0^1 \exp{s}ds=2e-2
\end{equation}

\subsubsection*{Work sheet}

\begin{enumerate}

\item The general second order Runge Kutta method for $\dot{y}=f(y)$ is
\begin{eqnarray}
k_1&=&f(y_n)\cr
k_2&=&f(y_n+\alpha \delta t k_1)
\end{eqnarray}
and
\begin{equation}
y_{n+1}=y_n+\left[\left(1-\frac{1}{2\alpha}\right)k_1+\frac{1}{2\alpha}k_2\right]\delta t
\end{equation}
Show that this gives the Taylor series up to second order. Note that
$\alpha=1/2$ gives the midpoint method. This was the last question on
last week's worksheet, so lots of people didn't get a chance to think
about it and it worth doing since it does give a good idea of how
Runge Kutta works.

\item Here is a second order differential equation
\begin{equation}
\frac{d^2f}{dt^2}+\frac{df}{dt}-6f=0
\end{equation}
This can also be solved using an ansatz of the form $A\exp{(\lambda
  t)}$; the difference is that there will be two different $\lambda$s, lets say $\lambda_1$ and $\lambda_2$. Since the equation is linear you can add them to give a solution with two arbitrary constants:
\begin{equation}
f(t)=A_1e^{\lambda_1t}+A_2e^{\lambda_2 t}
\end{equation}
This is what you expect for a second order differential equation, you
need two initial conditions, here use $f(0)=0$ and $\dot{f}(0)=-1$.

\item Here is another second order differential equation
\begin{equation}
\frac{d^2f}{dt^2}-f=0
\end{equation}
Solve this with $f(0)=0$ and $\dot{f}(0)=1$.

\item This differential equation
\begin{equation}
\frac{d^2f}{dt^2}+f=0
\end{equation}
with $f(0)=0$ and $\dot{f}(0)=1$ doesn't work so well, you end up with
complex $\lambda$s, however, if you keep your nerve and use the Euler formula
\begin{equation}
e^{i\theta}=\cos{\theta}+i\sin{\theta}
\end{equation}
it will work out; just bundle $A_1+A_2$ into one arbitrary constant
$C_1=A_1+A_2$ and $i(A_1-A_2)$ into another $C_2=i(A_1-A_2)$; $C_1$
and $C_2$ should turn out to be real, the detour through complex
numbers is just that, a detour.

\item Some integration examples; integrate
\begin{enumerate}
\item $\int (2x+2)e^{x^2+2x+3}dx$
\item $\int (x^2+1)/(x^3+3x)dx$
\item $\int \sqrt{x}dx$
\item $\int \sqrt{7x+1}dx$
\end{enumerate}

\item Integrating the square pulse. Say $f$ is square pulse
\begin{equation}
f(t)=\left\{\begin{array}{ll}1&-\pi/2<t<\pi/2\\0&\mbox{otherwise}\end{array}\right.
\end{equation}
Thus $f(t)$ is one between $-\pi/2$ and $\pi/2$ but zero everywhere else. What is
\begin{equation}
I=\int_{-\pi}^\pi f(t)\cos{t}dt
\end{equation}
what about
\begin{equation}
I=\int_{-\pi}^\pi f(t)\cos{nt}dt
\end{equation}
for $n$ an integer.
\end{enumerate}

\end{document}
