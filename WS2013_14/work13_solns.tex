
\documentclass[12pt]{article}

\usepackage{a4wide, amsfonts, epsfig}

\begin{document}
\begin{center}
{\bf EMAT10001 Workshop Sheet 13 outline solutions.}\\[1cm]{} Conor Houghton 2014-01-26
\end{center}

\subsubsection*{Work sheet}

\begin{enumerate}

\item Revise differentiating; find $df/dx$ for the following.
\begin{enumerate}
\item $f(x)=1/x$
\item $f(x)=(1+x)^3$, do this one both by using the chain rule and by multiplying it out.
\item $f(x)=A\exp(\lambda x)$ where $A$ and $\lambda$ are constants.
\item $f(x)=x\exp(x)$
\item $f(x)=\exp(x^2)$
\item $f(x)=\exp(1/x)$
\item $f(x)=x\exp(1+x)$
\end{enumerate}

\textbf{Solutions: } So $f(x)=1/x=x^{-1}$ then
$f'(x)=(-1)x^{-2}=-1/x^2$ where we are using the notation
$f'(x)=df(x)/dx$. For $(1+x)^3$ let $g=1+x$ so $dg/dx=1$ and
$dg^3/dg=3g^2$ giving
\begin{equation}
f'(x)=2(1+x)^2
\end{equation}
The other way
\begin{equation}
f(x)=(1+x)^3=1+3x+3x^2+x^3
\end{equation}
so
\begin{equation}
f'(x)=3+6x^2+3x^2=3(1+x)^2
\end{equation}
With $A\exp(\lambda x)$ we get $A\lambda \exp(\lambda x)$, we can do
this as a chain rule with $g(x)=\lambda x$. Next, for $f(x)=x\exp(x)$
using the product rule

\begin{equation}
f'(x)=e^x+xe^x
\end{equation}
For $f(x)=\exp(x^2)$ use the chain rule again to get $2xe^{x^2}$. Next $f(x)=\exp(1/x)$ then 
\begin{equation}
f'(x)=-\frac{e^{1/x}}{x^2}
\end{equation}
by the chain rule. Finally, using the chain rule and the product rule, if $f(x)=x\exp(1+x)$ then $f'(x)=\exp(1+x)+x\exp(1+x)$.

\item Using the power series, work out the value of $e$ to five decimal places.\\
\textbf{Solution: } Well $e=2.71828\ldots$, now the power series gives us $1$, $2$, $2.5$, $2.6666666\ldots$, $2.7083333\ldots$, $2.7166666\ldots$, $2.7180555\ldots$, $2.718243\ldots$, $2.7182787\ldots$, $2.718281\ldots$ and doing a few more to check it has stopped changing the first five places, we see it gets it after $n=9$.


\item By substituting in $f(t)=A\exp(\lambda t)$ solve 
\begin{equation}
\frac{d}{dt}f(t)=-2f(t)
\end{equation}
where $f(0)=5$. Substituting the $f(t)$ into the equation should give $\lambda$, fixing the initial condition, that is, the value of $f(0)$, should give $A$.\\
\textbf{Solution: } Well $df/dt=A\lambda \exp(\lambda t)$ and that has to be equal to $-2f=-2A\exp(\lambda t)$ which it is provided $\lambda=-2$ so
\begin{equation}
f(t)=Ae^{-2t}
\end{equation}
and now we have to fix $A$, if $f(0)=5$ and using $e^0=1$ we have $A=5$ and
\begin{equation}
f(t)=5e^{-2t}
\end{equation}


\item 
\item A breeding pair of rabbits are released on an island of endless
  grass. Each year each rabbit has five kits on average, these, in this
  approximation, grow up instantly. This rate is the average per rabbit,
  don't worry about the gender, the idea is that the doe have ten
  kits and the bucks none so the average is five. Thus, in this model
  if $P$ is the population of rabbits, the rate of increase is
  $dP/dt=5P$, ten kits for each member of the population and
  $P(0)=2$. What is the population after seven years.\\
\textbf{Solution: } So the equation is 
 \begin{equation}
\frac{dP}{dt}=10P
\end{equation}
and substituting $P=A\exp(\lambda t)$ as before we get $\lambda =10$,
if $P(0)=2$ this gives $A=2$ so
\begin{equation}
P(t)=2e^{5t}
\end{equation}
and 
\begin{equation}
P(7)=2e^{35}=3172026904626861
\end{equation}

\item Caesium-137 has a half-life of about 30 years; this means that a
  given Caesium-137 atom has a 0.0231 chance of decaying in a given
  year. To check this, write down a differential equation for the
  decay of Caesium-137 and solve it; check that the amount has halved
  after 30 years.\\
\textbf{Solution: } We have
\begin{equation}
\frac{dA}{dt}=-0.0231A
\end{equation}
where $A$ is the amount. Substituting in $C\exp{-rt}$ we find
\begin{equation}
A=A_0e^{-0.0231t}
\end{equation}
and if $t=30$ then $A=0.5A_0$.

\item The natural log is the inverse of the exponential:
\begin{equation}
\ln e^x=x
\end{equation}
Check this using your calculator for a couple of values of $x$.


\item Find a formula relating the decay constant and the half life;
  the decay constant is the chance of an individual atom decaying in a
  given period of time, the 0.0231 in the example above.\\
\textbf{Solution: } Well if we call the half-life $t_{0.5}$ then
\begin{equation}
0.5A_0=A_0e^{-rt_{0.5}}
\end{equation}
where $r$ is the decay constant. Now this means
\begin{equation}
2=e^{rt_{0.5}}
\end{equation}
and 
\begin{equation}
t_{0.5}=\frac{\ln{2}}{r}
\end{equation}


\item I discover a new craze, everyday, on average, I manage to
  persuade one new person to start in on my craze, they do the same
  and so on and so on. How long before everyone in the world shares my
  craze assuming the growth rate stays the same. Why is this assumption absurd?\\
\textbf{Solution: } Well the rate is one so everyone in the world has caught my craze at
\begin{equation}
7209000000=exp(t)
\end{equation}
so $t=\ln{7209000000}=22.7$ days. This is clearly absurd since as time passes it will be harder and harder to find people to pass the craze on to.


\item What is the $df/dx$ at $x=0$ for $f=\exp(-1/x)$, what about the second derivative 
\begin{equation}
\frac{d^2f}{dx^2}=\frac{d}{dx}\frac{df}{dx}
\end{equation}
at $x=0$. Can you guess what happens to higher derivatives at $x=0$?
To do this question you need to know that the exponential $\exp(-x)$
goes to zero as $x$ goes to infinity and, in fact, does so faster than
any polynomial!\\ 
\textbf{Solutions: } Well
\begin{equation}
f'(x)=\frac{e^{-1/x}}{x^2}
\end{equation}
and as $x\rightarrow 0$ we have $\exp{-1/x}\rightarrow 0$ and does so faster than $x^2$ goes to zero, so $f'(0)=0$. Finding the next derivative
\begin{equation}
f''(x)=-\frac{2}{x^3}e^{-1/x}-\frac{1}{x^3}e^{-1/x}
\end{equation}
so $f''(0)=0$ for the same reason. In fact, it is clear that the $n$th derivative looks like
\begin{equation}
f^{(n)}=\frac{\mbox{(some stuff)}e^{-1/x}}{\mbox{(other stuff)}}
\end{equation}
where all the \lq{}stuff\rq{} is polynomials in $x$ so $f^{(n)}(0)=0$
for all $n$. The point of all this is that $\exp(-1/x)$ doesn't have a
power series, the Taylor expansion, which we will look at later, is all zero.


\end{enumerate}

\end{document}
