\documentclass[12pt]{article}

\usepackage{a4wide, amsfonts, epsfig,bbold}

\begin{document}
\begin{center}
{\bf EMAT10001 Workshop Sheet 5.}\\[1cm]{} Conor Houghton 2013-10-17
\end{center}
\subsubsection*{Introduction} 
There is the usual bounty for errors and typos, 20p to \pounds 2
depending on how serious it is.

Some of these questions are taken from \emph{Number Theory with
  Computer Applications} by Ramanujachary Kumanduri and Cristina
Romero.

\subsubsection*{Useful facts}
\begin{itemize}
\item $a\equiv b \pmod m$ iff $m|(a-b)$.
\item $(a,m)=1$ if and only if there exists an $a^{-1}$ such that $aa^{-1}\equiv 1 \pmod m$. This can be found using the Euclid algorithm.
\item The Euclid algorithm is: set $x=a$ and $y=b$ where $a>b$, then, while $y!=0$
  set $r=x\bmod y$ and then let $x=y$ and $y=r$. If $y=0$ the answer
  is $x$.
\end{itemize}
\subsubsection*{Some common mathematical notation}
\begin{itemize}
\item ${\bf Z}$ or $\mathbb{Z}$, the integers, that is, whole numbers like zero, 67 and -120.
\item ${\bf N}$ or $\mathbb{N}$, the natural numbers are a subset of
  the integers. Unfortunately there is no universal agreement on
  whether they are the non-negative integers: $\{0,1,2,3,\ldots\}$ or
  the positive integers: $\{1,2,3,\ldots\}$.
\item ${\bf Q}$ or $\mathbb{Q}$, the rational numbers; there are numbers of the form $a/b$ where $a$ and $b\not=0$ are integers.
\item ${\bf R}$ or $\mathbb{R}$, the real numbers; these are the numbers used to measure continuous quantities.
\item $\forall$ for all, as in $x^2\ge 0\;\forall x\in {\bf Z}$.
\item $\exists$ exists, as in $\forall x\in {\bf N}\;\exists p\in {\bf N}$ with $p>x$ and $p$ a prime.
\item iff: if and only if. 
\end{itemize}


\subsubsection*{Work sheet}


\begin{enumerate}

\item Use the Euclid algorithm to find $x$ and $y$, integers, so that $9=945x+2421y$.

\item Prove that congruence is an equivalence relation. That is show
\begin{enumerate}
\item Reflexivity: $x\equiv x \pmod m$.
\item Symmetry: if $x\equiv y \pmod m$ then $y\equiv x \pmod m$.
\item Transitivity: if $x\equiv y \pmod m$ and
  $y\equiv z \pmod m$ then $x\equiv z \pmod m$.
\end{enumerate}
This is not a hard question, the proofs are only a line or so long.

\item Consider the set of all well-defined functions $f(x)$ where $x$
  is a real number. Now define $f(x)\sim g(x)$ if $f(0)=g(0)$. Prove
  this is an equivalence relation.

\item Prove the individual properties of congruences. If $a$, $b$,
  $c$, $d$ are integers and $m$ a positive integer then
\begin{enumerate}
\item $a\equiv b\pmod m$ implies $ac\equiv bc\pmod m$.
\item $a\equiv b\pmod m$ and $c\equiv d\pmod m$ implies $a+c\equiv b+d\pmod m$.
\item $a\equiv b\pmod m$ and $c\equiv d\pmod m$ implies $ac\equiv bd\pmod m$.
\item $a\equiv b\pmod m$ implies $a^k\equiv b^k\pmod m$ for positive integers $k$.
\item $a\equiv b\pmod m$ and $d|m$ for some $d$ a positive integer then $a\equiv b\pmod d$.
\item $ac\equiv bc\pmod m$ implies $a\equiv b\pmod {m/(c,m)}$.
\end{enumerate}
The last one is harder than the rest, start by letting $d=(c,m)$ and noting that this means $c=c'd$ and $m=m'd$.

\item Is -2 congruent to 31 modulo 11? 

\item Is 77 congruent to 5 modulo 12?

\item Find the inverse of 67 modulo 119 and the inverse of 119 modulo 67.

\item By looking at all the possibilities, show that 12 has no inverse modulo 18.

\item Solve $11x\equiv 28 \pmod {37}$.

\item Consider $(p-1)!=(p-1)\cdot (p-2)\ldots 1$ where $p$ is an odd
  prime. Now modulo $p$ only one and $p-1\equiv -1\pmod p$ are their
  own inverse; every other element has a unique inverse different from
  itself. Now by pairing each element with its inverse show that
  $(p-1)!\equiv -1 \pmod p$. This is known as Wilson's Theorem.

The next few problems are about cryptography, this is in preparation
for next week when we will look at RSA. The cryptographic schemes here
are simpler and don't really rely on the number theory we have been
doing.

\item The key idea behind cryptography is to keep messages secret. One
  of the most widely know cryptography techniques is called Caesar's
  cipher. The idea behind this is to shift all the letters a fixed
  amount down the alphabet. See if you can work out how this works and
  decipher these texts, each has a different shift.
\begin{enumerate}
\item Aqwtg iqppc pggf c dkiigt dqcv
\item Vlr hklt elt ql tefpqib, alkq vlr, Pqbsb? Vlr grpq mrq vlro ifmp qldbqebo xka yilt.
\item Rpyewpxpy, jzf nlye qtrse ty spcp. Estd td esp Hlc Czzx.
\end{enumerate}

\item One problem with Caesar's cipher is that you can just try every
  shift until you find one that makes sense. There are ways to make
  this harder, but many schemes are vulnerable to frequency analysis,
  for example, for Caesar's cipher, if you have lots of text you can
  just guess the most common letter is coding \lq{}e\rq{} and use that
  to calculate the shift and more complicated versions of this apply
  to more complicated ciphers. Vigen\`{e}re's cipher is designed to
  combat this. In this cipher you do modular arithmetic on letters, so
  for simplicity ignore the space and punctuation and number the
  letters zero through to 25. Now, to add two letters add the
  corresponding numbers modulo 26. Hence a+b=b, b+b=c and c+c=e. Now
  to use Vigen\`{e}re's cipher choose a code key, say
  \lq{}casablanca\rq{} and to encode \lq{}tomorrowisanotherday\rq{}
  you add c to t, a to o, s to m, a to o, b to r, l to r, a to o, n to
  w, c to i and a to s. At this points you have run out of letters in
  casablanca, so you start again, c to a, a to n and so on. Work out
  the Vigen\`{e}re cipher of \lq{}tomorrowisanotherday\rq{} using
  \lq{}casablanca\rq{} and the cipher of \lq{}bondjamesbond\rq{} using
  \lq{}drno\rq{}.

\item Decode \lq{}ihczsfmkoyysexgpwkqwmumiwrvlqeqa\rq{} with the key \lq{}maewest\rq{}.


\end{enumerate}

\subsubsection*{Exercise sheet}

The difference between the work sheet and the exercise sheet is that
the solutions to the exercise sheet won't be given and the problems
are designed to be more suited to working on on your own, though you
are free to discuss them in the work shop if you finish the work sheet
problems. Selected problems from the exercise sheet will be requested
as part of the continual assessment portfolio.

\begin{enumerate}

\item What is the inverse of 606 modulo 77? What is the inverse of 77 modulo 606?

\item Is 1111 congruent to 11 modulo 111?

\item Solve $42x\equiv 90 \pmod {156}$.

\item For the numbers from twenty to thirty say which are congruent to
  five modulo 13 and which are congruent to 13 modulo five.

\item Show that if $n$ is odd then $n^2\equiv 1\pmod 8$.

\item Let $p$ be an odd prime. Find the values of $x$ so that it is
  its own inverse modulo $p$.

\item Write a short program that allows you to input three numbers,
  $a$, $b$ and $c$ of modest size and tells you if $a\equiv b \pmod
  c$.

\item Write a program that tests if $a$ has an inverse modulo $m$ and
  which finds it if it does.

\item Write a program that automatically attempts to decode a passage
  encrypted using the Caesar cipher by assuming the most letter is 'e'.

\end{enumerate}


\subsubsection*{Challenge}
There is either a kitkat and copy of \emph{What is . . .} or a box of
chocolates, your choice, for the first person to solve
\texttt{projecteuler.net} problem 59. Provide proof by sending a
screen shot of the congratulations page, I will announce on the
website when the problem is solved. There is also a grand prize of a
blown, that is empty, ostrich egg for the first person to solve 25
\texttt{projecteuler.net} problems, prove it with a screen shot of the
progress page.

\end{document}
