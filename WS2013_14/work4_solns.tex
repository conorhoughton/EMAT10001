\documentclass[12pt]{article}
\usepackage{a4wide, amsfonts, epsfig}

\begin{document}
\begin{center}
{\bf EMAT10001 Workshop Sheet 4 - outline solutions.}\\[1cm]{} Conor Houghton 2013-10-17
\end{center}

\begin{enumerate}

\item No solution given.

\item If $r_1=a\bmod{c}$ then $a=n_1c+r_1$, similarly if
  $r_2=b\bmod{c}$ then $b=n_2c+r_2$ so the right hand side is
  $r_1+r_2\bmod{c}$ and the left hand side is
  $n_1c+r_1+n_2c+r_2\bmod{c}$ and we can loose the $(n_1+n_2)c$ since
  it's a multiple of $c$. For the second one the right hand side is
  $r_1r_2\bmod{c}$ whereas the left hand side is
  $(r_1+n_1c)(r_2+n_2c)\bmod{c}=(r_1r_2+\mbox{stuff}\cdot
  c)\bmod{c}=r_1r_2\bmod{c}$.

\item $60=3\cdot 4\cdot 5$ has 12 factors; it is the smallest number with 12 factors, but 72, 84, 90 and 96 do as well. The idea is to quicky write out the table, but you can guess it is a number with all the low factors, so, for example, under 1000 the best is 840=$3\cdot 5\cdot 7\cdot 8$ with 32 factors.

\item Take an example first, $8=2^3$ and it is divided by $1$, $2$,
  $4$ and $8$. More generally $p^r/p^s=p^{r-s}$ and the answer is
  $r+1$, taking in to account $s=0$.

\item The lemma about division.
\begin{enumerate}
\item If $a|b$ then $b=ma$ and if $x|y$ then $y=nx$ so $by=mnax$ and
  hence $ax|by$.
\item So if $a|b$ then $b=ma$ and if $b|c$ then $c=nb$ hence $c=mna$
  and $a|c$.
\item Well if $a|b$ then $b=na$ and $n\not=0$ if $b\not=0$ so $b>=a$.
\item In the usual way $b=ma$ and $c=na$ so $bx+cy=xma+yna$ so $a|(bx+cy)$.
\end{enumerate}

\item If one was a prime then the integers wouldn't be a unique factorization domain and lots of theorems would be harder to state.

\item The first 99 values are given at \texttt{http://oeis.org/A000010}.

\item If $d|p^n$ it must be in the form $p^s$, so if $d|a$ then $a=mp$
  for some $m$. Now, to work out the possible values of $m$, divide $p^n$ by $p$, giving $p^{r-1}$.

\item Any number that isn't $p$ is coprime with $p$ so $\phi(p)=p-1$.

\item From our calculation above, there are $p^{r-1}$ numbers which are co-prime with $p^r$ so 
\begin{equation}
\phi(p^r)=p^r-p^{r-1}=p^r\left(1-\frac{1}{p}\right)
\end{equation}

\item This follows from what we have done above
\begin{equation}
\phi(n)=n \prod \left(1-\frac{1}{p_i}\right) 
\end{equation}

\item So if $d|a$ and $d|b$ then $d|(a-b)$, conversely, if $d|(a-b)$
  and $d|b$ then $d|a$ so $a$ and $b$ and $a-b$ and $b$ have the same
  common divisors, so the have the same greatest divisor. The
  important point is you need to argue both ways to show the set of
  common divisors are the same and not just that one is contained in
  the other.

\item So the answer if $(\phi(n)-2)/2$; basically, if you skip on to
  the $i+k$th star each time, if $(n,k)\not=1$ you get back to where
  you started without visiting all the points, $k=1$ is explicitely
  excluded in the question and if $(n,k)=1$ then $(n-k,n)=1$ but that
  gives the same star, so you have to divide by two.

\item No answer given; you can check your answers at
  \texttt{http://gcd.awardspace.com/}, though of course, it is best to
  write your own program to do this.

\item It is useful to decide first of all what \lq{}slowest
  convergence\rq{} means; it should probably mean the most steps for
  the size of the number. Obviously then the slowest convergence will
  be numbers with gcd one, if $(a,b)=d$ then $(a/d,b/d)=1$ will take
  the same number of steps, but with $a/d$ and $b/d$ smaller than $a$
  and $b$ is $d\not=1$. Now, work backwards along the Euclid algorithm
  so that the numbers are as small as possible. The last step for
  numbers with gcd one is
\begin{equation}
x_2=k_1\cdot 1+0
\end{equation}
where I am calling the last number in the algorithm before you reach the one and zero $x_2$; the idea being that $x_1=1$ and $x_0=0$. Now to make $x_2$ as small as possible $k_1=1$ and $x_2=1$. Now go back one step to the step that lead to this one
\begin{equation}
x_3=k_2x_2+1
\end{equation}
and again for $x_3$ to be as small as possible $k_2=1$ giving $x_3=x_2+1=2$. Now keep going for as many terms as you are interested in, for the two numbers $x_n$ and $x_{n-1}$ to be as small as possible the $k_i$s are all chosen to be one so that
\begin{equation}
x_n=x_{n-1}+x_{n-2}
\end{equation}
and so on, that is, the Fibonacci sequence.

\item Symmetry and reflexivity are obvious, for transitivity $a\equiv
  b\pmod{n}$ means $a-b=m_1n$ for some $m_1$, similarly $b\equiv c\pmod{n}$ means $b-c=m_2n$ for some $m_2$, so $a-c=a-b+(b-c)=(m_1+m_2)n$ so $a\equiv c\pmod{n}$.

\end{enumerate}

\end{document}
