\documentclass[12pt]{article}
\usepackage{a4wide, amsfonts, epsfig}

\begin{document}
\begin{center}
{\bf EMAT10001 Workshop Sheet 1.}\\[1cm]{} 2 October 2013
\end{center}
\subsubsection*{Introduction} 
For this introductory week we will have some warm up fun with silly
logic puzzles. These puzzles are all taken from the book {\sl What is
  the name of this book} by {\bf Raymond M. Smullyan} and silly though
they are, solving them is good practice in logic.  

The teaching assistants are here to help, ask for help if you are
stuck, discuss the problems together, draw diagrams or truth tables on
the board. If you have solved a problem call over a teaching assistant
to explain your answer, he or she will pick a random student from the
table to give the explanation. The table that solves the most puzzles
wins.
\subsubsection*{Useful facts}
On this island {\bf knaves} always tell lies, {\bf knights} always
tell the truth. This means, for example, nobody on the island can say
\lq{}I am a knave\rq{} since that would be a lie if spoken by a knight and the
truth if spoken by a knave.

\subsubsection*{Questions}

\begin{enumerate}

\item Three islanders named Aoife, Brendan and Caoimhe, or any other
  three names starting with A, B and C you fancy, are standing
  together on a railway platform. A stranger passed by and asked
  Aoife, \lq{}Are you a knight or a knave?\rq{} She replied just as a
  train whistle blew and the stranger didn't hear her. The stranger
  then asked Brendan \lq{}What did Aoife say?\rq{} and Brendan replied 
  \lq{}Aoife said that she is a knave.\rq{} Caoimhe then says 
  \lq{}Don't believe Brendan, he is lying.\rq{} What are Brendan and
  Caoimhe?

\item In this problem there are only two people, Donnacha and
  Emer. Donnacha says \lq{}At least one of us is a knave.\rq{} What
  are Donnacha and Emer?

\item Suppose Donnacha had said \lq{}Either I am a knave or Emer is a
  knight\rq{} then what are Donnacha and Emer?

\item Again we have three islanders, Finn, Gavin and Hazel. Finn says 
  \lq{}All of us are knaves\rq and Gavin says \lq{}Exactly one of us
  is a knight\rq{}. What are Finn, Gavin and Hazel?

\item Suppose instead Finn had said \lq{}All of us are knaves\rq{} as
  before but Gavin had said \lq{}Exactly one of us is a knave.\rq{}
  Can it be determined what Gavin is? Can it be determined what Hazel
  is?

\item Two islanders this time: Iseult and Jarlath. Iseult says \lq{}I
  am a knave, but Jarlath isn't.\rq{} What are Iseult and Jarlath?

\item Three islanders again: Kieran, Liam and Maeve. Kieran says 
  \lq{}Liam is a knave\rq{} and Liam says \lq{}Kieran and Maeve are
  the same sort, both knaves or both knights.\rq{}. What is Maeve?

\item Three more islanders: Niamh, Orla and Padraig. Niamh says \lq{}Orla
  and Padraig are the same sort.\rq{} Someone then asks Padraig \lq{}Are Niamh
  and Orla the same sort?\rq{} What does Padraig answer?

\item For this puzzle there are three people, Quinn, Richard and Sorca, but
  this time we know one is a knave, one is a knight and the third is
  not from the island and can lie or tell the truth. Quinn says \lq{}I
  am not an islander\rq{} and Richard replies \lq{}That is true\rq{}. Sorca
  says \lq{}I am an islander\rq{}. What are Quinn, Richard and Sorca?

\item Now there are two people, Tadhg and Una but we don't know if
  they are islanders or not. Tadhg says \lq{}Una is a knight\rq{} and
  Una says \lq{}Tadhg is not a knight.\rq{} Prove that at least one of
  them is telling the truth, but is not a knight.

\item This next problem takes place on a different island where peoples behavior depends on the day of the week. Violet lies on Mondays, Tuesdays and Wednesdays, but tells the truth on the other days. Wendy lies on Thursdays, Fridays and Saturdays, but tells the truth on the other days. One day Xavier, who knew both well, met Violet and Wendy. Violet said \lq{}Yesterday was one of my lying days\rq{} and Wendy added \lq{}Yesterday was one of my lying days too\rq{}.  Which day is it?

\item Another day Xavier met Violet and she told him \lq{}I lied yesterday\rq{} and \lq{}I will lie again two days after tomorrow\rq{}. What day of the week is it?

\item On what days of the week can Violet say \lq{}I lied yesterday\rq{} and say \lq{}I will lie again tomorrow\rq{}?

\item On what days of the week can Violet say \lq{}I lied yesterday and I will lie again tomorrow\rq{}?

\item Now Yolanda and Zoe are two more inhabitants of this island, one is like Violet, lying on Mondays, Tuesdays and Wednesdays, the other is like Wendy, lying on Thursdays, Fridays and Saturdays. However, Xavier doesn't know which is which and, what's more, Yolanda and Zoe are twins who enjoy confusing people, so Xavier can't tell them apart. One day he meets them and one says \lq{}I am Yolanda\rq{} and the other says \lq{}I am Zoe\rq{}. Which one is really Yolanda and which is really Zoe?

\item Another day during the same week Xavier met the twins. One said
  \lq{}I'm Yolanda\rq{} and the said \lq{}If that's true, then I'm
  Zoe\rq{}. Which was which?

\end{enumerate}

\subsubsection*{End note}
{\it What is the name of this book?} carries on like this for two hundred pages, solving the final puzzle is equivalent to proving G\"{o}del's theorem, the famous mathematical proof that mathematics is incomplete. One of its later problems was adapted by the logician George Boolos and became known as {\bf the hardest logical puzzle ever}. It says
\begin{itemize}
\item Three gods A, B, and C are called, in no particular order, True,
  False, and Random. True always speaks truly, False always speaks
  falsely, but whether Random speaks truly or falsely is a completely
  random matter. Your task is to determine the identities of A, B, and
  C by asking three yes-no questions; each question must be put to
  exactly one god. The gods understand English, but will answer all
  questions in their own language, in which the words for yes and no
  are {\sl da} and {\sl ja}, in some order. You do not know which word means
  which.
\end{itemize}
See: {\tt http://en.wikipedia.org/wiki/The\_hardest\_logic\_puzzle\_ever}

\end{document}
