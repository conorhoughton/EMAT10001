\documentclass[12pt]{article}

\usepackage{a4wide, amsfonts, epsfig,bbold}

\begin{document}
\begin{center}
{\bf EMAT10001 Exercise Sheet 5.}\\[1cm]{} Conor Houghton 2013-10-17
\end{center}
\subsubsection*{Introduction} 
There is the usual bounty for errors and typos, 20p to \pounds 2
depending on how serious it is.

Some of these questions are taken from \emph{Number Theory with
  Computer Applications} by Ramanujachary Kumanduri and Cristina
Romero.

\subsubsection*{Useful facts}
\begin{itemize}
\item $a\equiv b \pmod m$ iff $m|(a-b)$.
\item $(a,m)=1$ if and only if there exists an $a^{-1}$ such that $aa^{-1}\equiv 1 \pmod m$. This can be found using the Euclid algorithm.
\item The Euclid algorithm is: set $x=a$ and $y=b$ where $a>b$, then, while $y!=0$
  set $r=x\bmod y$ and then let $x=y$ and $y=r$. If $y=0$ the answer
  is $x$.
\end{itemize}
\subsubsection*{Some common mathematical notation}
\begin{itemize}
\item ${\bf Z}$ or $\mathbb{Z}$, the integers, that is, whole numbers like zero, 67 and -120.
\item ${\bf N}$ or $\mathbb{N}$, the natural numbers are a subset of
  the integers. Unfortunately there is no universal agreement on
  whether they are the non-negative integers: $\{0,1,2,3,\ldots\}$ or
  the positive integers: $\{1,2,3,\ldots\}$.
\item ${\bf Q}$ or $\mathbb{Q}$, the rational numbers; there are numbers of the form $a/b$ where $a$ and $b\not=0$ are integers.
\item ${\bf R}$ or $\mathbb{R}$, the real numbers; these are the numbers used to measure continuous quantities.
\item $\forall$ for all, as in $x^2\ge 0\;\forall x\in {\bf Z}$.
\item $\exists$ exists, as in $\forall x\in {\bf N}\;\exists p\in {\bf N}$ with $p>x$ and $p$ a prime.
\item iff: if and only if. 
\end{itemize}


\subsubsection*{Exercise sheet}

The difference between the work sheet and the exercise sheet is that
the solutions to the exercise sheet won't be given and the problems
are designed to be more suited to working on on your own, though you
are free to discuss them in the work shop if you finish the work sheet
problems. Selected problems from the exercise sheet will be requested
as part of the continual assessment portfolio.

\begin{enumerate}

\item What is the inverse of 606 modulo 77? What is the inverse of 77 modulo 606?

\item Is 1111 congruent to 11 modulo 111?

\item Solve $42x\equiv 90 \pmod {156}$.

\item For the numbers from twenty to thirty say which are congruent to
  five modulo 13 and which are conguent to 13 modulo five.

\item Show that if $n$ is odd then $n^2\equiv 1\pmod 8$.

\item Let $p$ be an odd prime. Find the values of $x$ so that it is
  its own inverse modulo $p$.

\item Write a short program that allows you to input three numbers,
  $a$, $b$ and $c$ of modest size and tells you if $a\equiv b \pmod
  c$.

\item Write a program that tests if $a$ has an inverse modulo $m$ and
  which finds it if it does.

\item Write a program that automatically attempts to decode a passage
  encryped using the Caesar cipher by assuming the most letter is 'e'.

\end{enumerate}


\subsubsection*{Challenge}
There is either a kitkat and copy of \emph{What is . . .} or a box of
chocolates, your choice, for the first person to solve
\texttt{projecteuler.net} problem 59. Provide proof by sending a
screen shot of the congratulations page, I will announce on the
website when the problem is solved. There is also a grand prize of a
blown, that is empty, ostrich egg for the first person to solve 25
\texttt{projecteuler.net} problems, prove it with a screen shot of the
progress page.

\end{document}
