\documentclass[12pt]{article}

\usepackage{a4wide, amsfonts, epsfig}

\begin{document}
\begin{center}
{\bf EMAT10001 Workshop Sheet 16 outline solutions.}\\[1cm]{} Conor Houghton 2014-02-18
\end{center}

\subsubsection*{Work sheet}

\begin{enumerate}


\item Find $\partial f/\partial x$ and $\partial f/\partial y$ for 
\begin{enumerate}
\item $f(x,y)=xy\sin{xy}$
\item $f(x,y)=e^{x^2+y^2}$
\item $f(x,y)=xe^{xy}$
\item $f(x,y)=x^3+3x^2y+3xy^2+y^3$
\end{enumerate}

\textbf{Solutions: } So you just treat the other variables as
constants, for brevity $\partial_xf=\partial f/\partial x$ and the
same for $y$. This is a bad notation in every respect except being
shorter than all the good notations.
\begin{enumerate}
\item $\partial_xf=y\sin{xy}+xy^2\cos{xy}$ and $\partial_yf=x\sin{xy}+x^2y\cos{xy}$ 
\item $\partial_xf=2xe^{x^2+y^2}$ and $\partial_yf=2ye^{x^2+y^2}$.
\item $\partial_xf=e^{xy}+xye^{xy}$ and $\partial_yf=x^2e^{xy}$.
\item $\partial_xf=3x^2+6xy+3y^2$ and $\partial_yf=3x^2+6xy+3y^2$ 
\end{enumerate}
 
\item Find $\partial f/\partial x$, $\partial f/\partial y$ and $\partial f/\partial z$ for 
\begin{enumerate}
\item $f(x,y,z)=xy\ln{z}$
\item $f(x,y,z)=x^2+y^2+z^2$
\item $f(x,y,z)=x\sin{xyz}$
\end{enumerate}

\textbf{Solutions: }
\begin{enumerate}
\item $\partial_xf=y\ln{z}$, $\partial_yf=x\ln{z}$ and $\partial_zf=xy/z$
\item $\partial_xf=2x$ and so on.
\item $\partial_xf=\sin{xyz}+xyz\cos{xyz}$, $\partial_yf=x^2z\cos{xyz}$ and  $\partial_zf=x^2y\cos{xyz}$
\end{enumerate}


\item For $f(x,y)=x^3+3x^2y+3xy^2+y^3$ work out the directional
  derivative in the $(2,1)$ direction at $(1,0)$; don't forget to
  normalize the direction vector.

\textbf{Solution: } So
\begin{equation}
\nabla f=(3x^2+6xy+3y^2,3y^2+6xy+3x^2)
\end{equation}
which at $(1,0)$ is $\nabla f=(3,3)$
and since $|(2,1)|=\sqrt{5}$ we have $\mathbf{n}=(2/\sqrt{5},1/\sqrt{5})$ giving directional derivative $9/\sqrt{5}$.

\item Find the gradient of $f(x,y)=x+y^2$ and $f(x,y)=\sqrt{x^2+y^2}$.

\textbf{Solution: } grad$(x+y^2)=(1,2y)$ and 
\begin{equation}
\mbox{grad}\,\sqrt{x^2+y^2}=\left(\frac{x}{\sqrt{x^2+y^2}},\frac{y}{\sqrt{x^2+y^2}}\right)
\end{equation}

\item Going to three-dimensions in the obvious way, what is the
  gradient of 
\begin{equation}
f(x,y,z)=\sin{x}+\cos{y}+\sin{z}
\end{equation}
at $(\pi,0,\pi)$.

\textbf{Solution: } First the gradient
\begin{equation}
\nabla f=(\cos{x},-\sin{y},\cos{z})
\end{equation}
so at the point specified we have $(-1,0,-1)$.

\item The diverence is a differential operator acting on a vector
  field to give a scalar, that's the other way around to the gradient
  which acts on a scalar to give a vector field. It is defined by
  \begin{equation}
    \mbox{div}\,\mathbf{v}(x,y)=\mathbf{\nabla}\cdot\mathbf{v}=\frac{\partial v_1}{\partial x}+\frac{\partial v_2}{\partial y}
\end{equation}
What is the divergence of $(x,y)$? What about $(y,-x)$?

\textbf{Solution: }
\begin{equation}
\mbox{div\,}(x,y)=1+1=2
\end{equation}
and
\begin{equation}
\mbox{div\,}(y,-x)=0
\end{equation}

\item The Laplacian operator $\Box f=\mbox{div}(\mbox{grad}f)$. Write down the formula for $\Box f$ in terms of partial derivatives.

\textbf{Solution: } This comes from just writing it out, we are
looking for the divergence of $(\partial_x f,\partial_y f)$ so 
\begin{equation}
\Box f=\partial_x^2f+\partial_y^2f
\end{equation}
or in better notation
\begin{equation}
\Box f=\frac{\partial^2f}{\partial x^2}+\frac{\partial^2f}{\partial y^2}
\end{equation}

\item If a surface is given by $f(x,y,z)=c$ where $c$ is a constant,
  then grad$f$ is perpendicular to the surface. Examine the
  two-dimensional version by considering $x^2+y^2=1$. What is the
  gradient? On $x^2+y^2=1$ we can write $x=\cos{\theta}$ and
  $y=\sin{\theta}$ since these satisfy $x^2+y^2=1$. What does the
  gradient look like on the surface? Can you find a vector
  perpendicular to it, and therefore parallel to the surface.

\textbf{Solution: } This is harder to ask than answer. So 
\begin{equation}
\nabla(x^2+y^2)=(2x,2y)
\end{equation}
so on the circle we have $\mathbf{v}=(2\cos{\theta},2\sin{\theta})$. If we want something perpendicular to that we are looking for $\mathbf{w}$ such that $\mathbf{w}\cdot \mathbf{v}=0$, or
\begin{equation}
2w_1\cos{\theta}+2w_2\sin{\theta}=0
\end{equation}
and staring at that it is clear $(-\sin{\theta},\cos{\theta})$ works.


\end{enumerate}

 \end{document}
