\documentclass[12pt]{article}

\usepackage{a4wide, amsfonts, epsfig}

\begin{document}
\begin{center}
{\bf EMAT10001 Workshop Sheet 16.}\\[1cm]{} Conor Houghton 2014-02-18
\end{center}
\subsubsection*{Introduction} 
This worksheet is about multidimensional calculus and the gradient
vector. There is the usual bounty for errors and typos, 20p to \pounds
2 depending on how serious it is.

\subsubsection*{Useful facts}
\begin{itemize}
\item Partial derivative
\begin{eqnarray}
\frac{\partial f}{\partial x}&=&\lim_{h\rightarrow 0}\frac{f(x+h,y)-f(x,y)}{h}\cr
\frac{\partial f}{\partial y}&=&\lim_{h\rightarrow 0}\frac{f(x,y+h)-f(x,y)}{h}
\end{eqnarray}
In practise this means that to take the partial derivative with
respect to $x$ you treat $y$ like a constant and visa versa.
\item The gradient of $f(x,y)$
\begin{equation}
\mathbf{\nabla}f=\left(\frac{\partial f}{\partial x},\frac{\partial f}{\partial y}\right)
\end{equation}
\item Directional derivative
\begin{equation}
\nabla_{\mathbf{n}}f=\mathbf{\nabla}f\cdot \mathbf{n}
\end{equation}
\item To \textsl{normalize} a vector is to rescale it so it have
  length one.
\end{itemize}
\subsubsection*{Work sheet}

\begin{enumerate}


\item Find $\partial f/\partial x$ and $\partial f/\partial y$ for 
\begin{enumerate}
\item $f(x,y)=xy\sin{xy}$
\item $f(x,y)=e^{x^2+y^2}$
\item $f(x,y)=xe^{xy}$
\item $f(x,y)=x^3+3x^2y+3xy^2+y^3$
\end{enumerate}


\item Find $\partial f/\partial x$, $\partial f/\partial y$ and $\partial f/\partial z$ for 
\begin{enumerate}
\item $f(x,y,z)=xy\ln{z}$
\item $f(x,y,z)=x^2+y^2+z^2$
\item $f(x,y,z)=x\sin{xyz}$
\end{enumerate}

\item For $f(x,y)=x^3+3x^2y+3xy^2+y^3$ work out the directional
  derivative in the $(2,1)$ direction at $(1,0)$; don't forget to
  normalize the direction vector.

\item Find the gradient of $f(x,y)=x+y^2$ and $f(x,y)=\sqrt{x^2+y^2}$.

\item Going to three-dimensions in the obvious way, what is the
  gradient of 
\begin{equation}
f(x,y,z)=\sin{x}+\cos{y}+\sin{x}
\end{equation}
at $(\pi,0,\pi)$.

\item The diverence is a differential operator acting on a vector
  field to give a scalar, that's the other way around to the gradient
  which acts on a scalar to give a vector field. It is defined by
  \begin{equation}
    \mbox{div}\,\mathbf{v}(x,y)=\mathbf{\nabla}\cdot\mathbf{v}=\frac{\partial v_1}{\partial x}+\frac{\partial v_2}{\partial y}
\end{equation}
What is the divergence of $(x,y)$? What about $(y,-x)$?

\item The Laplacian operator $\Box f=\mbox{div}(\mbox{grad}f)$. Write down the formula for $\Box f$ in terms of partial derivatives.

\item If a surface is given by $f(x,y,z)=c$ where $c$ is a constant,
  then grad$f$ is perpendicular to the surface. Examine the
  two-dimensional version by considering $x^2+y^2=1$. What is the
  gradient? On $x^2+y^2=1$ we can write $x=\cos{\theta}$ and
  $y=\sin{\theta}$ since these satisfy $x^2+y^2=1$. What does the
  gradient look like on the surface? Can you find a vector
  perpendicular to it, and therefore parallel to the surface.

\end{enumerate}

\subsubsection*{Exercise sheet}

The difference between the work sheet and the exercise sheet is that
the solutions to the exercise sheet won't be given and the problems
are designed to be more suited to working on on your own, though you
are free to discuss them in the work shop if you finish the work sheet
problems. Selected problems from the exercise sheet will be requested
as part of the continual assessment portfolio.

\begin{enumerate}


\item Find $\partial f/\partial x$ and $\partial f/\partial y$ for 
\begin{enumerate}
\item $f(x,y)=xy\ln{xy}$
\item $f(x,y)=12x^4y+y^2$
\item $f(x,y)=xe^{y^2}$
\end{enumerate}


\item Find $\partial f/\partial x$, $\partial f/\partial y$ and $\partial f/\partial z$ for 
\begin{enumerate}
\item $f(x,y,z)=xyz$
\item $f(x,y,z)=(x-y)^2+(y-z)^2+(z-x)^2$
\end{enumerate}

\item For $f(x,y)=xy$ work out the directional
  derivative in the $(1,3)$ direction at $(1,1)$; don't forget to
  normalize the direction vector.

\item The third differential operator is curl; it acts on vector
  fields to give another vector field. It is only defined in three
  dimensions and has quite a complicated form
\begin{equation}
\mbox{curl}\mathbf{v}=\left(\frac{\partial v_3}{\partial y}-\frac{\partial v_2}{\partial z},\frac{\partial v_1}{\partial z}-\frac{\partial v_3}{\partial x},\frac{\partial v_2}{\partial x}-\frac{\partial v_1}{\partial y}\right)
\end{equation}
Show grad$\,(\mbox{curl}\mathbf{v})=0$.

\end{enumerate}

\subsubsection*{Challenge}
First four to email or tell me the answer:
\begin{eqnarray}
&&>+++++[<++>-]<+++ >++++[>+++<-]>[<++++>-]\cr 
&&<.+< . >. < . >-> >\cr
&&+++++>+< [ >[> >+>+< < <-] > > >[< < <+> > >-]< < < >[<+>-]\cr
&&< < < <[>+> > > > >+< < < < < <-] >[<+>-] > >\cr
&&[< <+> > > > >+< < <-] < <[> >+< <-]> > > > >.[-]\cr 
&&< < < < < < < .  > > > > > > [<+>-]< < <-]
\end{eqnarray}
The start of what?
 \end{document}
