\documentclass[12pt]{article}

\usepackage{a4wide, amsfonts, epsfig}

\begin{document}
\begin{center}
{\bf EMAT10001 Exercise Sheet 4.}\\[1cm]{} Conor Houghton 2013-10-17
\end{center}

\subsubsection*{Exercise sheet}

The difference between the work sheet and the exercise sheet is that
the solutions to the exercise sheet won't be given and the problems
are designed to be more suited to working on on your own, though you
are free to discuss them in the work shop if you finish the work sheet
problems. Selected problems from the exercise sheet will be requested
as part of the continual assessment portfolio.

\begin{enumerate}

\item Determine the set of integers for which the number of divisors is odd. Make a general conjecture and prove your claim.

\item $n!=n(n-1)(n-2)\ldots 2\cdot 1$ so $5!=5\cdot 4\cdot 3\cdot
  2\cdot 1=120$. What are the prime factors of $12!$.

\item If $a$ and $b$ are integers does $a^2|b^3$ imply $a|b$? Prove or disprove.

\item Show $2|(n^2-n)$.

\item Show the numbers $6k+5$ and $7k+6$ are co-prime for every $k\ge 1$.

\item Write a program to implement the Euclid algorithm.

\item Extend your Euclid algorithm  so that it returns $(a,b)=ma+nb$.

\item Write a program to calculate primes using the Sieve of Eratosthenes.

\item Write a program to find the Euler Totient of numbers of modest size. 

\item Imagine you wanted to calculate $a^b\bmod c$ for large values of
  $a$ and $b$. The straight-forward approach of calculating $a^b$ and
  then taking its mod is inefficient and will overwhelm data types
  like \texttt{int}. The usual approach is to write $b$ in the binary form
\begin{equation}
b=b_0+2b_1+4b_2+8b_3+\ldots
\end{equation}
with all the $b_i$s one or zero. Now $a^2\bmod c$ is easy to work out
by squaring and modding, squaring that $(a^2)^2\bmod c$ gives $a^4$
and so on. Use this to write a program to work out $a^b\bmod c$ which
will work provided $a^2$ fits into \texttt{int}.

\end{enumerate}


\subsubsection*{Challenge}

There are copies of \emph{What is the name of this book?} and a kitkat
available to the first person to solve each of the projecteuler.net
problems 3, 50, 69 and 214; that is, there are four prizes, one for each
problem. Provide proof by sending a screen shot of the congratulations
page, I will announce on the website when each of the problems is solved.

\end{document}
