\documentclass[12pt]{article}
\usepackage{a4wide, amsfonts, epsfig}

\begin{document}
\begin{center}
{\bf EMAT10001 Workshop Sheet 1 - outline solutions.}\\[1cm]{} 2 October 2013
\end{center}

These solutions are partly based on the solutions in {\sl What is the
  name of this book} by {\bf Raymond M. Smullyan} which is where the
problems came from. The usual bounty for errors applies, email me with
corrections.

\begin{enumerate}

\item No islander can say \lq{}I am a knave\rq{} so B must be lying
  and is therefore a knave. Since B is lying, C is telling the
  truth. C is therefore a knight. We don't know what A is and it isn't
  asked.

\item If D was a knave his statement would be true and that's a
  contradiction, thus D is a knight, his statement is true and E is a
  knave. [Notice that if V(D) means D is a knave and V(E)
  means E is a knave then !(V(D)$\|$V(E)) is !V(D)$\&$!V(E) where !
  means {\sl not}, $\|$ {\sl or} and $\&$ {\sl and}.]

\item For this to be lie then both parts must be false, that is D must
  be a knight and E must be a knave. However, if D is knight the
  statement must be true; this means D can't be a knave and the
  statement is true. Since D is a knight then E is also a
  knight. [Here we use two pieces of logic, first
    !(V(D)$\|$!V(E))=!V(D)$\&$V(E) and second, if P1$\|$P2 is true and
    P1 is false, then P2 must be true.

\item A knight could not say P=\lq{}All of us are knaves\rq{} so F is a
  knave. This means P must be false and there is at least one
  knight. If there are two knights G is a knight but Q=\lq{}Exactly
  one of us is a knight\rq{} is a lie, this is a contradiction, so
  there must be just one knight, Q is true, hence G is knight and so,
  by Q, H is a knave.

\item So, as before, F is a knave and there is at least one knight. If
  G is knight then his statement is true, so H is a knight. If G is
  knave then H must be a knight because we know there is at least
  one. This means we know H is a knight even though we don't know what
  G is.

\item I can't be a knight since a knight can't claim to be a
  knave. Since I is a knave then the overall statement must be false. Since
  the first part is true, the second part is false and so J is a knave.

\item If K is a knave then L is a knight and so M must be a knave, the
  same as K. On the other hand, if K is a knight then L is knave and M
  must be different from K and is therefore a knave.

\item For this one we have to chase down the different
  possibilities. If N is a knave then O and P are different. If P is a
  knave then O is a knight and so P lies and answers yes, if P is a
  knight then O is a knave and so P again answers yes, this time
  truthfully. On the other hand if N is a knight then O and P are the
  same and so P answers yes, truthfully if he is a knight, lyingly if
  he is a knave.

\item The answer given is yes or no and we are told the traveller is
  able to deduce the real answer from the reply. If the answer was yes
  then the traveller wouldn�t know what the real answer to the
  question; the one answering could be a knight, in which case the yes
  was truthfull, or he or she could be knave, in which case the yes is
  a lie. Since we are told the traveller learns the answer to the
  question yes can�t be the answer given. If the answer was no then
  the one who answered is a knave and the other a knight and the
  traveller learns the real answer is yes.

\item Q can't be a knight, a knight can't claim to be anything but a
  knight. If Q is not an islander then R's statement is true. Because
  Q is the person not from the island that means R is the knight and
  so S is the knave and that's a contradiction since S's statement
  would then be true. This means Q is the knave, R must be lying and
  so can't be the knight, he is therefore the person not from the
  island and S is the knight.

\item Obviously T and U can�t both be knights and so if U is a knight
  T is not, he is however telling the truth. If U is not a knight then
  then neither is T, which means U is telling the truth.

\item V can truthfully make the statement on Thursdays and make it as
  a lie on Mondays, W can truthfully make it on Sundays and as a lie
  on Thursdays, so it must be Thursday, the day they can both say \lq{}Yesterday was one of my lying days\rq{}.

\item Again, V can only say \lq{}I lied yesterday\rq{} on Mondays and
  Thursdays. If it is a Monday she is lying and so the second
  statement must be a lie, which it would be. If she says it on a
  Thursday she is telling the truth but the second statement would be
  lie and she can't lie, so it's a Monday.

\item She can never say that.

\item This is different, it is the whole statement that has to be true
  or false. Remember what�s more that !P(\&Q)=!P$\|$!Q. There is no day
  she can truthfully make this statement, so it must be a lie. On
  Monday the first part is false, on Wednesday the second part.

\item If the first person is Y, the second is Z and both are telling
  the truth, which can happen on a Sunday. If the first person is Z
  the second must be Y and so both are lying, there is no day both lie
  so this can�t happen.

\item Well now it isn't Sunday so one of the two statements must be
  false. If the first statement is true then the second is false and
  that leads to a contradiction. This means the first statement is
  false.

\end{enumerate}

\end{document}
