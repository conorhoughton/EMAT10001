\documentclass[12pt]{article}

\usepackage{a4wide, amsfonts, epsfig,bbold}

\begin{document}

\begin{center}
\textbf{EMAT10001 Workshop Sheet 7 outline solutions.}\\[1cm]{} Conor Houghton 2013-11-10
\end{center}


\subsubsection*{Work sheet}

\begin{enumerate}
\item Work out a multiplication table for $\{[1],[3],[5],[7]\}$ modulo eight. Show it is a group - don't worry about associativity, that follows from the associativity of modular multiplication. In fact, this is the other group with four elements; it is called the Klein four-group.\\
  \textbf{Solution: }
\begin{center}
\begin{tabular}{l|llll}
$\cdot $&$[1]$&$[3]$&$[5]$&$[7]$\cr
\hline
$[1]$&$[1]$&$[3]$&$[5]$&$[7]$\cr
$[3]$&$[3]$&$[1]$&$[7]$&$[5]$\cr
$[5]$&$[5]$&$[7]$&$[1]$&$[3]$\cr
$[7]$&$[7]$&$[5]$&$[3]$&$[1]$
\end{tabular}
\end{center}
and the key thing is that only $[1]$, $[3]$, $[5]$ and $[7]$ appear,
so it is closed, $[1]$ is an identity and the identity appears in
every row and column so this is a group.
\item Consider the digit 0, this has flipping over symmetries,
  flipping it horizonally $h$, flipping it vertically $v$ and doing
  both, one after the other, which we will write as $hv$. Write out a
  composition \lq{}after\rq{} table for this set of symmetries,
  including $e$, doing nothing. Is this group the Klein four-group
  $V_4$ or the four element group $\mathbf{Z}_4$ we saw in lectures.
  \textbf{Solution: } Checking $R_{180}=h\circ v$ well it gets flipped
  twice so it must be a rotation and it is easy to look at where
  specific points go. Now
\begin{center}
\begin{tabular}{l|llll}
$\circ $&$e$&$h$&$v$&$hv$\cr
\hline
$e$&$e$&$h$&$v$&$hv$\cr
$h$&$h$&$e$&$hv$&$v$\cr
$v$&$v$&$hv$&$e$&$h$\cr
$hv$&$hv$&$v$&$h$&$e$
\end{tabular}
\end{center}
To work this out just note that $h^2=v^2=(hv)^2=e$, everything is
Abelian so $vhv=hvv=hv^2=he=h$ and so on. Clearly closed and clearly
has an identity and, finally, each row and column has an identity in
it so there must be an inverse for every element. The big difference
between $V_4$ and $\textbf{Z}_4$ is that in the former all the
elements square to one, and that's true here, so this is $V_4$,
$[1]=[e]$, $[3]=h$, $[5]=v$ and $[7]=hv=[3][5]=[15]=[7]$ works.

\item Consider the addition table for $\{[0],[1],[2],[3]\}$ modulo four. Show this is a group -   don't worry about associativity, that follows from the associativity of modular addition. What is the identity? Is this group the Klein four-group $V_4$ or the four element group $\mathbf{Z}_4$ we saw in lectures.\\
  \textbf{Solution: } Now
\begin{center}
\begin{tabular}{l|llll}
$+ $&$[0]$&$[1]$&$[2]$&$[3]$\cr
\hline
$[0]$&$[0]$&$[1]$&$[2]$&$[3]$\cr
$[1]$&$[1]$&$[2]$&$[3]$&$[0]$\cr
$[2]$&$[2]$&$[3]$&$[0]$&$[1]$\cr
$[3]$&$[3]$&$[0]$&$[1]$&$[2]$
\end{tabular}
\end{center}
and this is a group with $[0]$ the identity. It is obviously $\textbf{Z}_4$ since $[1]+[1]=[2]$ so not all the elements are there own inverse as they are for the $V_4$ group. To find the isomorphism just make that up so in the lecture note the one that is its own inverse is $c$ and here it is $[2]$ so $[0]=e$ and $[2]=c$, the other two work either way.
\end{enumerate}

\end{document}
