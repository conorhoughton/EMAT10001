\documentclass[12pt]{article}

\usepackage{a4wide, amsfonts, epsfig}

\begin{document}
\begin{center}
{\bf EMAT10001 Exercise Sheet 13.}\\[1cm]{} Conor Houghton 2014-01-26
\end{center}

\subsubsection*{Exercise sheet}

The difference between the work sheet and the exercise sheet is that
the solutions to the exercise sheet won't be given and the problems
are designed to be more suited to working on on your own, though you
are free to discuss them in the work shop if you finish the work sheet
problems. Selected problems from the exercise sheet will be requested
as part of the continual assessment portfolio.

\begin{enumerate}

\item Solve 
\begin{equation}
\frac{df}{dt}=5f
\end{equation}
with $f(0)=12$.

\item Differentiate 
\begin{equation}
f(x)=x^3e^{x^3}
\end{equation}

\item Solve
\begin{equation}
\frac{df}{dt}=5(1-f)
\end{equation}
with $f(0)=0$.

\item The growth equation is not a realistic model of growth if there
  is a finite resource the population requires, this might be food, or
  space, or available uninfected individuals. The Verhulst-Pearl
  equation is an alternative that includes a \textsl{carrying
    capacity} for the environment, growth depends not only on the
  population but the residual carrying capacity. A simple
  Velhulst-Pearl equation is
\begin{equation}
\frac{dP}{dt}=P(1-P)
\end{equation}
More complicated versions include constants which have been set to one here. Solving this equation is tricky, it involves direct integration and a partial fractions expansion. However, it is easier to check the solution is indeed a solution, the solution with $P(0)=1/2$ is
\begin{equation}
P=\frac{1}{1+\exp(-t)}
\end{equation}
Check this.
\end{enumerate}

\subsubsection*{Challenge}
First three to get onto level five, that is complete four levels, of \texttt{http://www.pythonchallenge.com/} gets chocolate. Send a screenshot.

\end{document}
